% !Rnw weave = knitr
% !TeX program = pdflatex

%========================
% paper.tex
%========================
\documentclass[11pt]{article}\usepackage[]{graphicx}\usepackage[]{xcolor}
% maxwidth is the original width if it is less than linewidth
% otherwise use linewidth (to make sure the graphics do not exceed the margin)
\makeatletter
\def\maxwidth{ %
  \ifdim\Gin@nat@width>\linewidth
    \linewidth
  \else
    \Gin@nat@width
  \fi
}
\makeatother

\definecolor{fgcolor}{rgb}{0.345, 0.345, 0.345}
\newcommand{\hlnum}[1]{\textcolor[rgb]{0.686,0.059,0.569}{#1}}%
\newcommand{\hlsng}[1]{\textcolor[rgb]{0.192,0.494,0.8}{#1}}%
\newcommand{\hlcom}[1]{\textcolor[rgb]{0.678,0.584,0.686}{\textit{#1}}}%
\newcommand{\hlopt}[1]{\textcolor[rgb]{0,0,0}{#1}}%
\newcommand{\hldef}[1]{\textcolor[rgb]{0.345,0.345,0.345}{#1}}%
\newcommand{\hlkwa}[1]{\textcolor[rgb]{0.161,0.373,0.58}{\textbf{#1}}}%
\newcommand{\hlkwb}[1]{\textcolor[rgb]{0.69,0.353,0.396}{#1}}%
\newcommand{\hlkwc}[1]{\textcolor[rgb]{0.333,0.667,0.333}{#1}}%
\newcommand{\hlkwd}[1]{\textcolor[rgb]{0.737,0.353,0.396}{\textbf{#1}}}%
\let\hlipl\hlkwb

\usepackage{framed}
\makeatletter
\newenvironment{kframe}{%
 \def\at@end@of@kframe{}%
 \ifinner\ifhmode%
  \def\at@end@of@kframe{\end{minipage}}%
  \begin{minipage}{\columnwidth}%
 \fi\fi%
 \def\FrameCommand##1{\hskip\@totalleftmargin \hskip-\fboxsep
 \colorbox{shadecolor}{##1}\hskip-\fboxsep
     % There is no \\@totalrightmargin, so:
     \hskip-\linewidth \hskip-\@totalleftmargin \hskip\columnwidth}%
 \MakeFramed {\advance\hsize-\width
   \@totalleftmargin\z@ \linewidth\hsize
   \@setminipage}}%
 {\par\unskip\endMakeFramed%
 \at@end@of@kframe}
\makeatother

\definecolor{shadecolor}{rgb}{.97, .97, .97}
\definecolor{messagecolor}{rgb}{0, 0, 0}
\definecolor{warningcolor}{rgb}{1, 0, 1}
\definecolor{errorcolor}{rgb}{1, 0, 0}
\newenvironment{knitrout}{}{} % an empty environment to be redefined in TeX

\usepackage{alltt}

%---- Page & fonts
\usepackage[margin=1in]{geometry}
\usepackage[T1]{fontenc}
\usepackage[utf8]{inputenc}
\DeclareUnicodeCharacter{2013}{--}   % en dash
\DeclareUnicodeCharacter{2014}{---}  % em dash
\DeclareUnicodeCharacter{2212}{-}    % unicode minus
\usepackage{lmodern}
\usepackage[american]{babel}
\usepackage{microtype}
\usepackage{authblk}

%---- Math & symbols
\usepackage{amsmath, amssymb, mathtools}
\usepackage{bbm}          % for \mathbbm{1} indicator
\usepackage{siunitx}
\usepackage{dcolumn}   % needed for stargazer's D{.}{.}{-3} columns
\usepackage{csquotes}  % recommended with biblatex (silences that warning)

%---- Graphics, tables, floats
\usepackage{graphicx}
\graphicspath{{fig/}{figures/}{build/}} % adjust as needed
\usepackage{booktabs, threeparttable}
\usepackage{caption}
\usepackage{subcaption}
\usepackage{float}
\usepackage{placeins}

%---- % Allows abstract customization
\usepackage{abstract} 
\renewcommand{\abstractnamefont}{\normalfont\bfseries} % Set the "Abstract" text to bold
%\renewcommand{\abstracttextfont}{\normalfont\small\itshape} % Set the abstract itself to small italic text



%---- Links & clever refs
\usepackage[hidelinks]{hyperref}
\usepackage[capitalize,noabbrev]{cleveref}

%---- Citations & bib
\usepackage[backend=biber,style=authoryear,dashed=false,doi=false,isbn=false,url=false,arxiv=false]{biblatex}
\addbibresource{/Users/hectorbahamonde/Bibliografia_PoliSci/library.bib}  % your .bib file


%---- Custom macros (edit as you like)
\newcommand{\govdist}{\texttt{Government distance}}
\newcommand{\govclose}{\texttt{gov\_closeness\_w\_01}}
\newcommand{\techno}{\texttt{Technocracy}}
\newcommand{\demsat}{\texttt{Q8\_1}}      % democratic satisfaction
\newcommand{\trustpol}{\texttt{Q9\_4}}    % trust in Finnish politicians
\newcommand{\ind}[1]{\mathbbm{1}\!\left\{#1\right\}}

%\title{Losers, Delegation, and the Ideological Valence of Expertise: Evidence from Finland}
\vspace{-1cm}\title{\textbf{\input{title.txt}\unskip}} % Article title





\author[1]{\textsc{Katri Aaltonen}\thanks{\href{mailto:katri.m.aaltonen@utu.fi}{katri.m.aaltonen@utu.fi}; \href{https://www.utu.fi/fi/ihmiset/katri-aaltonen}{\texttt{https://www.utu.fi/fi/ihmiset/katri-aaltonen}}}}
\author[2]{\textsc{Hector Bahamonde}\thanks{\href{mailto:hector.bahamonde@utu.fi}{hector.bahamonde@utu.fi}; \href{https://www.hectorbahamonde.com}{\texttt{https://www.hectorbahamonde.com}}}}
\author[3]{\textsc{Mikko Niemelä}\thanks{\href{mailto:mikko.niemela@utu.fi}{mikko.niemela@utu.fi}; \href{https://www.utu.fi/fi/ihmiset/mikko-niemela}{\texttt{https://www.utu.fi/fi/ihmiset/mikko-niemela}}. \\{\bf Authors are listed in alphabetical order; all authors contributed equally.}}}


\affil[1]{Academy Research Fellow, INVEST Research Flagship Centre, University of Turku, Finland}
\affil[2]{Senior Researcher, INVEST Research Flagship Centre, University of Turku, Finland}
\affil[3]{Professor of Sociology, INVEST Research Flagship Centre, University of Turku, Finland}

%\affil[ ]{\textit{Authors are listed in alphabetical order; all authors contributed equally.}}




\date{\today}
\IfFileExists{upquote.sty}{\usepackage{upquote}}{}
\begin{document}


\maketitle
\thispagestyle{empty}


\subsection*{1. Motivation and Research Question}

Healthcare systems in wealthy democracies confront an acute and growing dilemma: the life-saving or life-extending promise of innovative cancer medicines collides with their extraordinary costs, creating tragic choices that pit individual rescue against collective stewardship of finite resources. When a patient with incurable cancer has exhausted conventional treatment options and a novel but vastly expensive medicine offers only an uncertain possibility of modest health benefit, decision-makers at every level---from individual clinicians to pharmaceutical regulators to policymakers to the general public---must navigate irreducible moral terrain. \textcite[1216]{Torkki2022} document that ``medicine costs have increased rapidly (37-125\%) in all countries'' across the Nordic region between 2012 and 2017, driven substantially by expensive cancer treatments with uncertain clinical impact.

Cancer medicines represent a uniquely challenging case for understanding public opinion on healthcare rationing for several reasons. First, cancer is universally recognized as a serious and often fatal disease, activating powerful emotional and moral responses. Second, many novel cancer medicines offer only modest improvements in survival or quality of life, creating genuine scientific uncertainty about their value. \textcite[1]{Aziz2020} show that ``adding atezolizumab to nab-paclitaxel resulted in an additional 0.361 QALYs (0.636 LYs) at an ICER of S\$324,550 per QALY gained,'' illustrating the dramatic cost-to-benefit ratios that define this domain. Third, cancer patients are typically identifiable---specific individuals with known prognoses and names---rather than statistical abstractions, triggering what healthcare ethics scholars call the ``rule of rescue.'' Fourth, the innovation pipeline ensures that cancer drug costs will continue rising, making the policy question not merely contemporary but structurally enduring.

Yet little is known about how citizens in democratic welfare states evaluate these tragic trade-offs. Do publics apply the cost-effectiveness reasoning that health economists and policy analysts advocate? Or do citizens' preferences for funding expensive cancer medicines depend systematically on whether the decision context activates moral imperatives (rescue an identifiable patient) versus fiscal imperatives (steward finite healthcare resources)?

This article addresses a central research question: **How do citizens evaluate healthcare trade-offs between expensive, uncertain cancer treatments and healthcare system sustainability when policy choices are framed as loss avoidance (rescue of identifiable patients) versus efficiency-oriented resource allocation (protection of broader healthcare provision)?** We theorize that citizens' support for expensive cancer medicines is not primarily determined by objective information about costs or clinical benefits, but rather by whether the decision context activates a loss-prevention mindset (wherein withholding treatment appears to incur an avoidable loss of life) or a gains-protection mindset (wherein funding the treatment appears to threaten other healthcare provision). By anchoring public healthcare preferences in prospect theory---a psychological theory of decision-making under risk that predicts asymmetric responses to losses and gains---we argue that framing effects represent not mere rhetorical noise but fundamental drivers of healthcare policy preferences.

\subsection*{2. Literature Review}

\subsubsection*{2.1 The Healthcare Context: Cancer Medicines and the Rationing Dilemma}

Modern cancer medicine exemplifies the central tension in contemporary healthcare systems: scientific innovation creates therapeutic possibilities while economic constraints limit access. The problem is both clinically and politically acute. \textcite[1216]{Torkki2022} find that across Nordic countries, ``cancer care in Nordic countries has significant differences in both cost structures and in the development of cost drivers, indicating differences in the organization of care and different focus in health policy.'' More dramatically, \textcite[2]{Nguyen2025} show that in the United States, ``very high-cost users lived in areas with higher social needs'' and that ``human immunodeficiency virus, inflammatory conditions, multiple sclerosis, and cancer accounted for the largest share of costs'' among these populations.

The financial pressure is relentless. \textcite[851]{Torkki2022} report that ``healthcare costs of cancer in real terms increased in all countries: CAGR was between 1 and 6\% depending on the country,'' yet more concerning is that ``medicine costs have increased rapidly (37-125\%) in all countries during the observation period.'' This explosive growth in pharmaceutical costs creates a fundamental mismatch: healthcare budgets are finite, yet cancer drug prices increase exponentially.

Compounding this cost problem is profound clinical uncertainty. Unlike older cancer treatments, many novel medicines offer only incremental improvements in survival or quality of life, making their therapeutic value debatable. \textcite[851]{Aziz2020} demonstrate this concretely: ``osimertinib is not cost effective as a first-line treatment compared to standard EGFR TKIs in advanced EGFR mutant NSCLC patients in Singapore,'' with cost-effectiveness ratios far exceeding standard willingness-to-pay thresholds. Yet such analyses, which guide policy decisions in many countries, often fail to convince patient advocates or the general public that withholding treatment is justified.

This gap between health economic analysis and public sentiment reflects deeper disagreements about what should matter in healthcare allocation decisions. Health technology assessment (HTA) bodies globally have developed frameworks to evaluate expensive medicines, yet \textcite[42]{Ng2024} note that ``high-income countries reviewed have implemented a variety of pathways and mechanisms for reimbursing high-cost medicines under specific eligibility criteria, listing processes, varying cost-effectiveness thresholds and special funding arrangements,'' indicating that technical analysis alone does not determine policy. The question becomes: how do citizens navigate the gap between cost-effectiveness analysis and moral obligations to treat individual patients?

\subsubsection*{2.2 Public Health Deliberation on Cancer Drugs and Values}

When given opportunity for deliberation, citizens reveal nuanced value systems for evaluating expensive cancer medicines. \textcite[1]{Bentley2018} conducted a deliberative engagement with British Columbia citizens on cancer drug funding and found that ``participants accepted the principle of resource scarcity and the need of governments to make difficult trade-offs when allocating health-care resources. They supported the view that cost-benefit thresholds must be set for high-cost drugs. They also expected reasonable health benefits in return for large expenditures, and supported the view that some drugs do not merit funding.'' 

More importantly, \textcite[1]{Bentley2018} identify what citizens actually value: ``Participants accepted the principle of resource scarcity and the need of governments to make difficult trade-offs...they also wanted drug funding decisions to be made in a non-partisan and transparent way.'' This suggests that citizens can support rationing decisions if they perceive the process as fair and evidence-based.

Yet this deliberative acceptance of rationing disappears in other contexts. \textcite[2-3]{Noh2025} surveyed South Korean public opinion and found that ``three-quarters of the respondents agreed or strongly agreed that high-cost cancer drugs should be reimbursed,'' yet this support was deeply conditional: ``however, they support reimbursement only when they are confident that the drug is safe, effective, and cost-effective.'' Critically, \textcite[3]{Noh2025} identify a key tension: ``the South Korean public perceives the reimbursement of high-cost cancer drugs as both a human rights measure and a government responsibility,'' framing expensive cancer treatment not as a luxury but as a fundamental entitlement.

This reframing of expensive cancer treatment from policy question to human rights issue is consequential. \textcite[831]{Smith2025a} find that in Ghana, public support for health taxes ``increased when health taxes were framed as measures to improve public health and/or create a fairer tax system,'' indicating that citizens' willingness to support healthcare spending depends heavily on how the rationale is presented.

A critical question emerges from this literature: what causes the gap between citizens' abstract acceptance of rationing principles and their concrete support for expensive medicines? One answer involves the salience of identifiable patients. \textcite[1]{Jones2022a} argue that healthcare policy should include ``a wider range of values in healthcare policy,'' noting that ``public value evaluation proceeds through values inquiry, establishing what is important to different stakeholders (including policy makers, healthcare staff, patients and communities).'' When citizens encounter specific, identifiable cancer patients---as they do in media coverage, patient advocacy, and clinical narratives---different moral frameworks become activated than when they consider abstract resource allocation principles.

\subsubsection*{2.3 Framing Effects in Health Information and Policy}

How health information is presented fundamentally shapes how citizens and professionals interpret that information. The foundational medical demonstration comes from Tversky and Kahneman's work on medical decision-making. When describing identical survival probabilities for surgery versus radiation in lung cancer treatment, \textcite[341]{Kahneman1984a} found that ``preferences of physicians and patients between hypothetical therapies for lung cancer varied markedly when their probable outcomes were described in terms of mortality or survival.'' Specifically, when the same data were described in terms of deaths rather than survivals, surgery became less attractive despite identical underlying probabilities---the frame, not the facts, determined preference.

This framing effect extends beyond clinical contexts to health policy. \textcite[9]{Roh2016} conducted experiments on framing effects in health policy preferences regarding sugar-sweetened beverages, finding that ``strong liberals had more support for policies designed to reduce the consumption of these drinks when the policies referenced soda compared to sugar-sweetened beverage,'' while ``strong conservatives showed the opposite pattern.'' Importantly, \textcite[9]{Roh2016} demonstrate that the same policy generates different support depending on ``the images returned from the search query,'' suggesting that frames operate at multiple cognitive levels, shaping both conscious reasoning and automatic associations.

In the specific context of expensive medicines, framing matters profoundly. \textcite[5]{McCabe2023} examined public support for government intervention in unexpected medical bills and found substantial variation depending on how the problem was framed. When bills were presented as failures of specific providers or insurers, respondents blamed those entities and sought their direct regulation; when presented as system-level problems, respondents supported broader healthcare reform. This suggests that citizens' policy preferences are not fixed but contingent on how decision problems are presented.

Critically, \textcite[1411-1430]{Valenzuela2021} show in their analysis of COVID-19 framing effects that ``competing frames and melodrama'' shape policy preferences differently than health economists might predict. Specifically, \textcite[1420]{Valenzuela2021} find that ``participants exposed to Facebook posts with an economic frame were significantly less supportive of measures that restrict mobility,'' yet surprisingly, ``exposure to a public health frame also reduced support for stay-at-home orders,'' contrary to expectations. This counter-intuitive finding suggests that frame effects in health policy are complex and non-linear, mediated by emotions, prior beliefs, and social factors.

\subsubsection*{2.4 Rule of Rescue in Medical Contexts and Healthcare Ethics}

A distinctive feature of healthcare decision-making is the power of the ``rule of rescue''---the moral intuition that when a specific, identifiable individual faces certain death and rescue is possible, that rescue becomes imperative regardless of cost. \textcite[2407]{McKie2003} define the rule of rescue as ``the imperative people feel to rescue identifiable individuals facing avoidable death,'' and argue that ``Jonsen coined the term `Rule of Rescue' to describe the imperative people feel to rescue identifiable individuals facing avoidable death. In this paper we attempt to draw a more detailed picture of the RR, identifying its conflict with cost-effectiveness analysis, the preference it entails for identifiable over statistical lives.''

This phenomenon manifests starkly in healthcare. \textcite[2407]{McKie2003} explain that the rule of rescue ``entails for identifiable over statistical lives, the shock-horror response it elicits, the preference it entails for lifesaving over non-lifesaving measures.'' Clinically, this appears when a single patient with a devastating diagnosis receives extensive resources despite poor prognosis, whereas preventive programs serving thousands receive minimal funding---a pattern that violates cost-effectiveness logic yet reflects deep moral commitments.

Healthcare professionals face this dilemma acutely. \textcite[1-2]{Bashkin2022} conducted interviews with oncologists, patients, and family members and found that ``the economic consideration in the decision on cancer treatment'' emerged as a core theme, yet ``most patients expect'' their physicians to focus on treatment possibilities rather than economic constraints. Critically, \textcite[1-2]{Bashkin2022} identify ``psychosocial aspects of the discussion on treatment costs and health policy,'' suggesting that cost discussions in cancer care activate emotional and moral frameworks beyond economic calculation.

The rule of rescue creates a persistent tension in healthcare policy. \textcite[2407-2408]{McKie2003} note that while rescue logic can be ``defended from a utilitarian point of view, on the ground that rescues increase well-being by reinforcing people's belief that they live in a community that places great value upon life,'' this justification fails on fairness grounds: ``fairness requires that we do not discriminate between individuals on morally irrelevant grounds, whereas being `identifiable' does not seem to be a morally relevant ground for discrimination.''

Yet the rule of rescue persists because of deeper temporal and narrative considerations. \textcite[33]{Sinclair2022a} argues that rescue obligations stem not primarily from identifiability but from temporal proximity and narrative closure: ``We are particularly averse to depriving people of the opportunity to follow some such pattern as they approach death. This means allowing them to sort out their affairs, say goodbyes to family and friends, review their life, or come to terms with death itself.'' In this framing, rescue of an identified cancer patient reflects commitment to dignified death rather than simple identifiability bias.

\subsubsection*{2.5 Comparative Evidence on Reimbursement Decisions and Public Opinion}

Globally, countries struggle with how to fund expensive cancer medicines while maintaining healthcare system sustainability. \textcite[3]{Scheijmans2025} surveyed Dutch citizens on reimbursement of expensive medicines and found that ``although a majority considered the CL policy unjustified, they preferred it to the alternative policy measures that were presented,'' indicating that citizens, when forced to choose among imperfect options, prefer structured rationing to ad hoc decisions. In four real-life case descriptions, \textcite[3]{Scheijmans2025} found that ``respondents most often indicated effectiveness, lack of availability of alternative treatment and improved quality of life due to treatment as reasons for a positive reimbursement decision. An unfavourable cost-benefit ratio was their main reason to be against reimbursement.''

This suggests that citizens apply multi-criteria reasoning in healthcare decisions---effectiveness, alternatives, quality of life, and cost all matter---yet the weight assigned to each criterion may shift depending on frame. When the frame emphasizes a patient's last resort (no alternatives), citizens weight effectiveness differently than when the frame emphasizes system sustainability.

\subsection*{3. Literature Gap}

Three critical gaps structure the current research landscape. 

First, most experimental studies of framing effects in healthcare have relied on non-representative populations (typically college students), limiting generalizability to mass public opinion. The few exceptions---notably deliberative studies like Bentley et al.'s work with British Columbia citizens---employ small samples and extended engagement that may not reflect how citizens process health policy information in typical survey contexts.

Second, while prospect theory has been extensively applied to political decision-making and welfare-state policy reform, its application to public preferences regarding expensive healthcare interventions remains underdeveloped. \textcite[4]{Hameleers2020} provide one exception, applying prospect theory to COVID-19 pandemic framing, finding that ``gain frames of the coronavirus promote support for risk-aversive interventions, whereas loss frames result in more support for risk-seeking alternatives.'' Yet this work examines pandemic responses, not the specific domain of expensive chronic disease treatments.

Third, and most critically, the literature has not integrated two key insights: (1) that healthcare decisions activate distinctive moral frameworks around identifiable patients and rescue, and (2) that these moral frameworks interact with prospect-theoretic reference points to shape risk preferences. The existing literature either emphasizes healthcare ethics and rescue (treating moral frameworks as normative questions) or applies prospect theory to policy domains (treating moral frameworks as secondary to reference points). What is missing is an integrated framework showing how prospect theory's predictions about gain/loss domains interact with healthcare-specific moral commitments.

This gap is consequential because it leaves policymakers without theoretical guidance for a practical problem: citizens simultaneously endorse rationing in principle and expensive treatments in practice. An account that merely emphasizes either rationality failures or moral commitments cannot explain this apparent contradiction. Prospect theory offers such an account: the contradiction dissolves when we recognize that different frames activate different reference points, which in turn activate different moral frameworks and risk orientations.

\subsection*{4. Theory: Prospect Theory and Healthcare Policy Preferences}

\subsubsection*{4.1 Foundational Principles of Prospect Theory}

Prospect theory begins from an observation that contradicts standard economic theory: when decision-makers evaluate options under risk, they treat losses and gains asymmetrically. \textcite[263]{Kahneman1979} demonstrate that ``the psychological principles that govern the perception of decision problems and the evaluation of probabilities and outcomes produce predictable shifts of preference when the same problem is framed in different ways.''

Central to the theory is loss aversion---the empirical finding that losses hurt more than equivalent gains please. \textcite[193]{Kahneman1991a} explain that ``loss aversion---the disutility of giving up an object is greater that the utility associated with acquiring it.'' More formally, \textcite[73]{K.R2024} define loss aversion as ``a cognitive bias where individuals are strongly motivated to avoid losses or psychologically, they perceive loss is more severe than an equivalent gain.''

Critically, individuals evaluate outcomes not in absolute terms but relative to a reference point. \textcite[334]{Vis2011} explain that ``a principal feature of prospect theory is that it posits that individuals' risk tendency varies across contexts, with individuals being risk averse in the domain of gains and risk accepting in the domain of losses.'' The reference point---usually the status quo---determines whether a given outcome counts as gain or loss. \textcite[334]{Vis2011} note that ``individuals use a reference point, usually the status quo, to establish whether they find themselves in a situation or domain of losses or of gains'' and that empirically, ``losses weigh typically two to two and a half times more heavily than gains.''

\subsubsection*{4.2 Reference Points in Healthcare: What Counts as Gain Versus Loss?}

Healthcare decisions activate distinctive reference points that differ from typical policy domains. In health contexts, the reference point is not merely the current state of affairs but the expected health trajectory absent intervention. This creates several distinctive patterns.

For an identifiable cancer patient with poor prognosis, the reference point shifts from ``current impaired health'' to ``imminent death or severe deterioration.'' Relative to this reference point, any treatment offering even uncertain possibility of benefit represents potential gain (extended life, quality of life improvement). The frame emphasizes what will be lost---the patient's life---if treatment is withheld.

By contrast, for the healthcare system and taxpayers, the reference point is ``current healthcare provision and financial stability.'' Relative to this reference point, expensive new medicine threatens existing provision, representing a potential loss (foregone resources for other patients, reduced system sustainability). The frame emphasizes what will be lost---capacity to serve other patients---if resources are diverted to expensive treatments.

These reference points are not objectively given; they are constituted by how the decision problem is framed. When a cancer patient's case is presented as ``a patient has exhausted all standard treatments and faces imminent death,'' the natural reference point is that patient's expected death. When the same case is presented as ``healthcare budgets are finite,'' the reference point becomes system-level sustainability. Prospect theory predicts that reference point shifts will alter risk preferences systematically.

\subsubsection*{4.3 Loss Frame in Healthcare: Rule of Rescue and Risk-Seeking}

When healthcare decisions are framed in terms of potential loss---specifically, the loss of an identifiable patient's life or health if treatment is withheld---prospect theory predicts increased risk-seeking behavior. An identifiable cancer patient whose only remaining option is an expensive, uncertain medicine represents precisely this scenario: the reference point is the patient's certain death (or severe deterioration) absent treatment, making treatment an attempt to avoid catastrophic loss.

Under loss framing in healthcare, \textcite[290]{McDermott2004} explains that ``leaders in a bad situation, where things are bad or likely to get worse, are more likely to make risky choices to recover their losses.'' In healthcare terms, a patient in a bad situation (advanced cancer with no alternatives) is predicted to accept risky treatments. Extending this logic, citizens asked whether expensive, uncertain cancer treatment should be funded, when the frame emphasizes a specific patient's imminent death, are also predicted to become risk-seeking---willing to accept financial risk and clinical uncertainty to attempt to prevent that death.

The rule of rescue operates precisely through this mechanism. \textcite[2407]{McKie2003} explain that ``when public decision-making is structured so that a decision focuses on a specific identifiable victim rather than on aggregate or statistical victims, substantial resources are sometimes devoted to rescue.'' Prospect theory explains why: because identifiable victims are presented as facing certain bad outcomes (death), the decision is framed as a loss domain, activating risk-seeking preferences.

Clinically, this manifests as the dominance of treatment imperatives over cost considerations. \textcite[1]{Bashkin2022} find that when oncologists and patients discuss innovative cancer therapies, ``patients expect'' their physicians to explore treatment possibilities, reflecting the implicit framing of cancer as a loss domain where treatment represents an opportunity to avoid catastrophic loss. The emotional weight of an identified patient's mortality activates what \textcite[33]{Sinclair2022a} calls the imperative to allow people ``to sort out their affairs, say goodbyes to family and friends, review their life, or come to terms with death itself''---framed as potential losses that treatment can prevent.

\subsubsection*{4.4 Gain Frame in Healthcare: Resource Stewardship and Risk Aversion}

By contrast, when healthcare decisions are framed in terms of utility maximization and finite resources, prospect theory predicts risk-averse behavior. The reference point shifts from an individual patient's expected death to the healthcare system's capacity to serve all patients. Relative to this reference point, approving expensive treatment is construed as a threat---a potential loss of resources that could serve other patients.

Under gain-framed or neutral healthcare decisions, \textcite[334]{Vis2011} predict that individuals ``being risk averse in the domain of gains.'' In healthcare policy terms, when the frame emphasizes budgetary sustainability and opportunity costs---``the funds used for this medicine will not be available elsewhere in healthcare''---citizens are predicted to prefer safer, lower-risk options: rejecting unconditional funding or demanding cost reductions.

This prediction finds support in the empirical health policy literature. \textcite[3]{Scheijmans2025} find that when Dutch citizens are presented with policy constraints---``there are limited resources for healthcare''---they apply stricter standards: ``an unfavourable cost-benefit ratio was their main reason to be against reimbursement.'' The frame highlighting resource scarcity activates a gains-domain reference point (protecting existing provision) rather than a loss-domain reference point (preventing patient death).

Similarly, in climate policy, \textcite[1]{Svenningsen2021} demonstrate that ``a gain and loss framing influence social preferences for the distributional outcomes of climate policy,'' with loss frames (framing inaction as loss of future income) generating higher willingness to pay for policy than gain frames. Extending this finding to healthcare: when resource constraints are framed neutrally or as gains (``we can maintain healthcare for all patients''), citizens become risk-averse in their demands for expensive treatments.

\subsubsection*{4.5 Health Outcomes Under Uncertainty: The Reference Point Problem}

A distinctive feature of healthcare decisions is that the clinical outcome itself is uncertain. Unlike many policy domains where gains and losses are clearly defined, healthcare involves uncertain health outcomes, creating what might be called a ``nested reference point problem.''

For an expensive cancer medicine with uncertain efficacy, citizens must evaluate not merely the cost (clear loss) versus potential treatment benefit (uncertain), but must also assess the clinical evidence itself. When a medicine has uncertain effectiveness---as \textcite[1]{Aziz2020} demonstrates for many cancer treatments---citizens must establish a reference point for what counts as meaningful health benefit.

Prospect theory's prediction is that under uncertainty about health benefits, citizens in loss domains (identifiable patient facing death) will become more accepting of risk, willing to try treatments with low probability of benefit if there is any chance of preventing catastrophic health loss. By contrast, in gains domains (healthy system maintenance), citizens will demand higher certainty of benefit before accepting the cost.

This creates a systematic pattern: the same medicine with the same uncertain efficacy will be supported in loss-framed contexts (identifiable patient, imminent mortality) and opposed in gain-framed contexts (resource stewardship, system sustainability). The objective facts about the medicine do not change; only the reference point changes, systematically altering how citizens evaluate the risk-benefit trade-off.

\subsubsection*{4.6 Integration: Why Cost-Effectiveness Often Fails and Rescue Often Succeeds}

Cost-effectiveness analysis (CEA) applies a gains-frame logic: it values healthcare resources and asks whether the benefit gained (measured in QALYs, life-years, etc.) justifies the cost. This analysis necessarily adopts a system-level reference point (what is the best use of limited healthcare resources?). \textcite[851]{Aziz2020} exemplify this approach: ``adding atezolizumab to nab-paclitaxel resulted in an additional 0.361 QALYs at an ICER of S\$324,550 per QALY gained,'' leading to a policy recommendation that the medicine is not cost-effective.

But this CEA recommendation implicitly asks: should we allocate resources to this treatment, or to something else? This frames the decision in a gains domain: we have resources, and we should allocate them optimally. In this frame, prospect theory predicts risk aversion: citizens and policymakers will prefer lower-cost, more certain options.

Yet when the same medicine is presented in a loss frame---``a specific patient has cancer, all standard treatments have failed, this medicine offers a small chance of benefit, without it the patient will die''---the decision is reframed as loss avoidance, not resource allocation. The reference point becomes the patient's certain death, not the healthcare system's resource envelope. In this frame, prospect theory predicts risk-seeking: citizens will support trying the medicine, accepting both the financial risk and the clinical uncertainty.

This explains a persistent puzzle in healthcare policy: why do patients, patient advocates, and often the public reject cost-effectiveness arguments even when those arguments are scientifically sound? The answer is not that the public is irrational, but that cost-effectiveness analysis implicitly assumes a gains-frame reference point (optimal resource allocation), while rescue narratives activate a loss-frame reference point (preventing patient death). When citizens evaluate healthcare decisions relative to a patient's mortality, cost-effectiveness calculations recede in importance.

\subsubsection*{4.7 Health-Specific Implications: Beyond Individual Versus Population}

The application of prospect theory to healthcare decisions reveals why the individual rescue versus population health tension is not a simple matter of competing values, but reflects systematic differences in how reference points are framed.

Individual rescue cases (identifiable patient, imminent death, experimental treatment) activate a loss-frame reference point: the patient's expected death. Citizens in this frame become risk-seeking, willing to accept low-probability treatments. Population health approaches (preventive medicine, health system sustainability) activate a gains-frame reference point: maintaining health system capacity. Citizens in this frame become risk-averse, demanding higher certainty of benefit.

Neither frame is objectively correct. A healthcare system must simultaneously rescue individuals and maintain population health. But prospect theory predicts that citizens' preferences for each will shift depending on which reference point is made salient through framing. \textcite[290]{McDermott2004} note that ``prospect theory offers a number of advantages that justify the use of psychological models over alternative models of political behavior'' because it ``emphasizes the importance of loss in calculations of value and utility,'' revealing why loss frames (rescue, individual mortality) are psychologically more compelling than gains frames (system sustainability, resource optimization).

The implication is that healthcare policy decisions should not be made purely on the basis of cost-effectiveness analysis (which assumes gains-frame reasoning) or purely on rescue logic (which assumes loss-frame reasoning), but should explicitly acknowledge how framing shapes public preferences. Citizens' apparent inconsistency---supporting both fiscal responsibility and expensive treatments---is actually systematic inconsistency generated by shifting reference points.

\subsection*{5. Methods Overview}

\subsubsection*{5.1 Data Source and Sampling}

The data analyzed in this study originate from the Finnish Medicines Barometer, a national, cross-sectional population survey administered biennially by the Finnish Medicines Agency to examine experiences, opinions, and values related to health, medicines, and well-being. The 2021 wave included an ad hoc module containing a survey experiment specifically designed to measure public opinions regarding public funding of novel, expensive oncology medicines with uncertain clinical benefit.

Finland provides an apt research context for several reasons. First, Finland maintains a comprehensive public healthcare system with universal coverage, similar to most wealthy democracies, making findings generalizable to other welfare-state contexts. Second, \textcite[1216-1222]{Torkki2022} document that Nordic countries including Finland face precisely the problem this study addresses: ``cancer care in Nordic countries has significant differences in both cost structures and in the development of cost drivers,'' driven substantially by expensive pharmaceutical innovations.

Data collection was conducted through a market research company (Taloustutkimus Ltd). Survey respondents were drawn from a pre-recruited internet panel containing approximately 40,000 Finnish citizens. To achieve the target sample size of 2,000 respondents, 10,105 invitations were distributed to panelists. The panel was stratified to ensure representation across gender, age, education level, and geographic region, with the target population defined as Finnish-speaking citizens aged 18--79 years. The final sample included 2,081 respondents, yielding a completion rate of approximately 20.6\%.

\subsubsection*{5.2 Experimental Design and Clinical Case Description}

The experimental module included two distinct survey experiments: an information experiment and a framing experiment. This article focuses exclusively on results from the framing experiment, which employed a randomized between-subjects design with three treatment conditions (Scenarios B, C, and D), each administered to approximately 500 respondents.

All three scenarios presented respondents with an identical case description and decision-making task. Critically, the case description provided genuine clinical information about expensive cancer medicine, grounding the hypothetical in realistic health scenarios. The case description introduced a new medicine indicated for treatment of a specific incurable cancer. The description specified that annually 10 to 20 Finnish patients would qualify for the medicine; that while the medicine does not cure the disease, laboratory results indicate it destroys cancer cells in approximately one-third of patients who receive it; that it remains unknown whether the medicine extends patients' lives or improves quality of life relative to other available treatments; and that the medicine carries significant risk of adverse effects. These clinical features---uncertain effectiveness, high toxicity potential, small patient population---mirror real cancer medicine scenarios.

Crucially, the case specified that if approved for public funding in Finland, the medicine's cost would exceed 60,000 euros per patient at the final cancer treatment stage. This cost figure, which \textcite[1]{Aziz2020} show is realistic for innovative cancer treatments, anchors respondents to the genuine fiscal implications of expensive cancer medicine decisions.

After presenting this case, the experimental conditions diverged through the introduction of alternative frames administered before the decision-making task. These frames were designed to manipulate the reference point as predicted by prospect theory.

**Scenario B** (baseline/control condition) presented the case description without additional framing, establishing a control against which framing effects could be measured. Respondents in this condition received only the clinical and cost information.

**Scenario C** (rescue/loss frame) introduced framing designed to activate a loss-domain reference point by emphasizing patient mortality and treatment scarcity: ``There is no cure for this particular type of cancer. The new medicine is a possible option for patients who have already received multiple treatments and for whom the remaining options are limited.'' This frame explicitly makes salient: (1) the incurable nature of the cancer, (2) the patient's exhaustion of standard options, and (3) the scarcity of remaining treatment possibilities. Relative to these reference points, any medicine offers potential rescue from certain death.

**Scenario D** (utility-maximization/gains frame) introduced framing designed to activate a gains-domain reference point by emphasizing budgetary constraints and opportunity costs: ``The funds available for healthcare are finite. The adoption of the new medicine means that the funds used to pay for it will not be available elsewhere in healthcare.'' This frame explicitly makes salient: (1) the limited healthcare budget, and (2) the zero-sum nature of funding decisions. Relative to this reference point, expensive treatment threatens existing provision.

These frames are theoretically motivated by prospect theory: the rescue frame should activate loss-domain reasoning (patient will die without treatment), while the utility-maximization frame should activate gains-domain reasoning (resources will be lost to other uses if spent on this treatment). The frames are also realistic---they reflect how healthcare decisions are actually presented in policy debates and media coverage.

\subsubsection*{5.3 Dependent Variable and Response Options}

Following the case description and frame manipulation (where applicable), all respondents answered an identical decision-making task: ``What kind of decision regarding the new medicine's use would you find acceptable? Please choose the option that best reflects your opinion.'' Four response options were provided, ordered to reflect increasing risk-aversion in healthcare spending:

(1) \textit{Risk-seeking, unconditional funding}: ``The medicine should be adopted to use and be paid for with society's money regardless of the price set on it by the pharmaceutical company''

(2) \textit{Conditional funding with price negotiation}: ``The medicine should be adopted to use and be paid for with society's money if the pharmaceutical company lowers its price''

(3) \textit{Risk-averse, no funding}: ``The medicine should not be adopted to use and be paid for with society's money''

(4) \textit{Uncertain}: ``I don't know''

These response options permit measurement of risk preferences in healthcare spending. Option 1 represents maximum risk-seeking (accepting high cost, uncertain efficacy), option 2 represents moderate risk-aversion (accepting treatment only if cost is reduced), and option 3 represents maximum risk-aversion (rejecting treatment entirely). 

Prospect theory predictions are clear: the rescue frame (Scenario C) should increase support for option 1 (unconditional funding) and decrease support for option 3 (no funding), reflecting shift toward risk-seeking in the loss domain. The utility-maximization frame (Scenario D) should decrease support for option 1 and increase support for option 2 or 3 (conditional or no funding), reflecting shift toward risk-aversion in the gains domain.

\subsection*{6. Contribution}

This study advances research on public opinion and healthcare policy in six substantive ways. 

First, it provides evidence from a representative population sample regarding the magnitude and direction of framing effects on public support for expensive cancer medicines---a policy domain where framing effects have been theoretically predicted but empirically understudied. While laboratory studies have demonstrated framing effects in abstract choice tasks, and deliberative studies have shown citizens can reason about healthcare trade-offs, few studies have experimentally tested framing effects on representative public samples evaluating consequential healthcare decisions.

Second, it demonstrates the utility of prospect theory as an explanatory framework for understanding welfare-state public opinion in healthcare specifically, showing that the same citizens can hold apparently contradictory preferences (fiscal responsibility and rescue logic) when different reference points are activated by framing. This challenges utility-maximizing models that assume consistent preferences across contexts.

Third, it advances healthcare ethics by providing a psychological mechanism for understanding the rule of rescue. Rather than treating rescue as normatively justified or empirically irrational, prospect theory shows it as a systematic consequence of how loss-frame reference points activate risk-seeking preferences. This bridges the gap between healthcare ethics literature (which emphasizes moral imperatives) and behavioral economics (which emphasizes systematic deviation from rationality).

Fourth, it offers practical guidance for healthcare policymakers regarding communication about expensive medicines. If framing systematically affects public support, then transparent discussion of how decisions are framed becomes essential to healthcare legitimacy. Policy decisions presented only in cost-effectiveness terms (gains frames) may lose public legitimacy that could be maintained through balanced presentation of both individual rescue imperatives and system sustainability concerns.

Fifth, it contributes to the growing literature on how prospect theory applies to collective healthcare decisions. While prospect theory was developed for individual choice, this study extends it to public policy preferences, showing that the theory predicts preference reversals for groups and populations, not merely individuals.

Sixth, and most importantly, it advances integration of health-specific decision-making with prospect-theoretic insights. Rather than applying prospect theory to healthcare as though it were any other policy domain, this work shows how health-specific factors---identifiable mortality, clinical uncertainty, distinctively moral frameworks around rescue---interact with prospect theory's reference-point-dependent preferences to generate predictable but non-obvious patterns of public opinion.


















%\newpage
\vspace{-1cm}
\begin{abstract}
\input{abstract.txt}\unskip
\end{abstract}



\vspace*{0.3cm}
\centerline{{\bf Abstract length}: 2 words.}
\vspace*{0.3cm}



%\centerline{\bf Please consider downloading the last version of the paper \href{https://raw.githubusercontent.com/hbahamonde/democratic_backsliding/main/2025/Dem_Backsliding_2.pdf}{\texttt{{\color{red}here}}}.}

\vspace*{0.3cm}
\centerline{\bf {\color{red}PLEASE DO NOT CIRCULATE}.}

\vspace*{0.5cm}
\centerline{\providecommand{\keywords}[1]{\textbf{\textit{Keywords---}} #1} % keywords.  
\keywords{{\input{keywords.txt}\unskip}}}



\clearpage
\pagenumbering{arabic}
\setcounter{page}{1}

\section{Public Support for Healthcare Policies: Is the glass half-full or half-empty?}


\FloatBarrier
\newpage



\clearpage
\newpage
\pagenumbering{roman}
\setcounter{page}{1}
\printbibliography
\clearpage
\newpage



%%%%%%%%%%%%%%%%%%%%%%%%%%%%%%%%%%%%%%%%%%%%%%
% WORD COUNT
%%%%%%%%%%%%%%%%%%%%%%%%%%%%%%%%%%%%%%%%%%%%%%
\clearpage



\begin{center}
\vspace*{\stretch{1}}
\dotfill
\dotfill {\huge {\bf Word count}: 5,831} \dotfill
\dotfill
\vspace*{\stretch{1}}
\end{center}

\clearpage

%%%%%%%%%%%%%%%%%%%%%%%%%%%%%%%%%%%%%%%%%%%%%%
% WORD COUNT
%%%%%%%%%%%%%%%%%%%%%%%%%%%%%%%%%%%%%%%%%%%%%%


% Online Appendix
%\newpage
%\section{Online Appendix}
%\pagenumbering{Roman}
%\setcounter{page}{1}



%% reset tables and figures counter
\setcounter{table}{0}
\renewcommand{\thetable}{A\arabic{table}}
\setcounter{figure}{0}
\renewcommand{\thefigure}{A\arabic{figure}}



\section{Appendix}
\pagenumbering{roman}
\setcounter{page}{1}


\subsection{Regression Models}
\FloatBarrier

Here.

\end{document}

By applying a theory of losses to the study of public support for healthcare policies, our argument is twofold. First, individuals are more likely to support expansive public spending on high-cost medicines when the situation is framed as a potential loss—especially when it threatens the life of patients. In these contexts, citizens tend to overcompensate to avoid perceived harm, even if that harm is probabilistic or affects a small group. Rather than prioritizing cost-effectiveness, we find that citizens are disproportionately motivated by the idea of preventing tragic, losses. This pattern reverses the traditional assumption that citizens, when confronted with trade-offs, act as rational utility-maximizers. In fact, existing research supports this intuition. For instance, Slovic (2007) and Jenni and Loewenstein (1997) have shown that people are far more willing to take costly actions to save specific, identified individuals than when the beneficiaries are anonymous or statistical. Why would the public endorse unconditional funding for a treatment that benefits only a handful of patients and threatens healthcare budgets elsewhere? Our explanation is rooted in how people respond to potential loss: when exposed to “rescue” narratives, citizens perceive the withholding of care as morally unacceptable and react with disproportionate support for public provision—even at great cost. 

Second, this tendency intensifies when the perceived moral stakes are high, or when the treatment is framed as a last resort. In these cases, the public is not merely insensitive to cost—they become actively risk-seeking, favoring funding decisions that would otherwise be seen as inefficient or unsustainable. The idea of “not doing everything possible” feels like an intolerable failure. Thus, our findings diverge from traditional public policy models that assume preferences are grounded in budgetary realism. Instead, we argue that citizens are psychologically committed to avoiding loss—particularly avoidable death or suffering—even when it entails greater risk. This aligns with previous work suggesting that moralized framing can dominate cost-benefit logic (Kogut \& Ritov 2005; Baron 1997). Importantly, our findings are not simply about emotional response—they reflect a broader pattern in which loss sensitivity reshapes how people prioritize fairness, need, and resource allocation in health policy. In these contexts, supporting expensive treatment is not about maximizing public utility—it is about not being complicit in loss. 

 

Here’s some for your consideration and to be freely modified /Katri:  

Data collection:  

  

Data originates from the Finnish Medicines Barometer, a national, cross-sectional population survey collected by the Finnish Medicines Agency biennially, to examine experiences, opinions, and values related to health, medicines, and well-being. The 2021 wave included an ad hoc module with a survey experiment on opinions related to public funding on novel, expensive oncology medicines. Data collection was purchased from a market research company (Taloustutkimus Ltd). Survey respondents were derived from a pre-recruited internet panel containing approximately 40,000 Finns. To achieve the target number of 2,000 respondents, 10,105 invitations were sent to panelists representing 18−79-year-old Finnish-speaking Finns, stratified by gender, age, level of education and area of residence, yielding 2,081 responses. 

