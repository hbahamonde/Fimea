% !Rnw weave = knitr
% !TeX program = pdflatex

%========================
% paper.tex
%========================
\documentclass[11pt]{article}\usepackage[]{graphicx}\usepackage[]{xcolor}
% maxwidth is the original width if it is less than linewidth
% otherwise use linewidth (to make sure the graphics do not exceed the margin)
\makeatletter
\def\maxwidth{ %
  \ifdim\Gin@nat@width>\linewidth
    \linewidth
  \else
    \Gin@nat@width
  \fi
}
\makeatother

\definecolor{fgcolor}{rgb}{0.345, 0.345, 0.345}
\newcommand{\hlnum}[1]{\textcolor[rgb]{0.686,0.059,0.569}{#1}}%
\newcommand{\hlsng}[1]{\textcolor[rgb]{0.192,0.494,0.8}{#1}}%
\newcommand{\hlcom}[1]{\textcolor[rgb]{0.678,0.584,0.686}{\textit{#1}}}%
\newcommand{\hlopt}[1]{\textcolor[rgb]{0,0,0}{#1}}%
\newcommand{\hldef}[1]{\textcolor[rgb]{0.345,0.345,0.345}{#1}}%
\newcommand{\hlkwa}[1]{\textcolor[rgb]{0.161,0.373,0.58}{\textbf{#1}}}%
\newcommand{\hlkwb}[1]{\textcolor[rgb]{0.69,0.353,0.396}{#1}}%
\newcommand{\hlkwc}[1]{\textcolor[rgb]{0.333,0.667,0.333}{#1}}%
\newcommand{\hlkwd}[1]{\textcolor[rgb]{0.737,0.353,0.396}{\textbf{#1}}}%
\let\hlipl\hlkwb

\usepackage{framed}
\makeatletter
\newenvironment{kframe}{%
 \def\at@end@of@kframe{}%
 \ifinner\ifhmode%
  \def\at@end@of@kframe{\end{minipage}}%
  \begin{minipage}{\columnwidth}%
 \fi\fi%
 \def\FrameCommand##1{\hskip\@totalleftmargin \hskip-\fboxsep
 \colorbox{shadecolor}{##1}\hskip-\fboxsep
     % There is no \\@totalrightmargin, so:
     \hskip-\linewidth \hskip-\@totalleftmargin \hskip\columnwidth}%
 \MakeFramed {\advance\hsize-\width
   \@totalleftmargin\z@ \linewidth\hsize
   \@setminipage}}%
 {\par\unskip\endMakeFramed%
 \at@end@of@kframe}
\makeatother

\definecolor{shadecolor}{rgb}{.97, .97, .97}
\definecolor{messagecolor}{rgb}{0, 0, 0}
\definecolor{warningcolor}{rgb}{1, 0, 1}
\definecolor{errorcolor}{rgb}{1, 0, 0}
\newenvironment{knitrout}{}{} % an empty environment to be redefined in TeX

\usepackage{alltt}

%---- Page & fonts
\usepackage[margin=1in]{geometry}
\usepackage[T1]{fontenc}
\usepackage[utf8]{inputenc}
\DeclareUnicodeCharacter{2013}{--}   % en dash
\DeclareUnicodeCharacter{2014}{---}  % em dash
\DeclareUnicodeCharacter{2212}{-}    % unicode minus
\usepackage{lmodern}
\usepackage[american]{babel}
\usepackage{microtype}
\usepackage{authblk}

%---- Math & symbols
\usepackage{amsmath, amssymb, mathtools}
\usepackage{bbm}          % for \mathbbm{1} indicator
\usepackage{siunitx}
\usepackage{dcolumn}   % needed for stargazer's D{.}{.}{-3} columns
\usepackage{csquotes}  % recommended with biblatex (silences that warning)

%---- Graphics, tables, floats
\usepackage{graphicx}
\graphicspath{{fig/}{figures/}{build/}} % adjust as needed
\usepackage{booktabs, threeparttable}
\usepackage{caption}
\usepackage{subcaption}
\usepackage{float}
\usepackage{placeins}

\usepackage{tabularray}
\usepackage{codehigh}
\usepackage[normalem]{ulem}
\UseTblrLibrary{booktabs}
\UseTblrLibrary{siunitx}
\newcommand{\tinytableTabularrayUnderline}[1]{\underline{#1}}
\newcommand{\tinytableTabularrayStrikeout}[1]{\sout{#1}}
\NewTableCommand{\tinytableDefineColor}[3]{\definecolor{#1}{#2}{#3}}


%---- % Allows abstract customization
\usepackage{abstract} 
\renewcommand{\abstractnamefont}{\normalfont\bfseries} % Set the "Abstract" text to bold
%\renewcommand{\abstracttextfont}{\normalfont\small\itshape} % Set the abstract itself to small italic text



%---- Links & clever refs
\usepackage[hidelinks]{hyperref}
\usepackage[capitalize,noabbrev]{cleveref}

%---- Citations & bib
\usepackage[backend=biber,style=authoryear,dashed=false,doi=false,isbn=false,url=false,arxiv=false]{biblatex}
\addbibresource{/Users/hectorbahamonde/Bibliografia_PoliSci/library.bib}  % your .bib file


%---- Custom macros (edit as you like)
\newcommand{\govdist}{\texttt{Government distance}}
\newcommand{\govclose}{\texttt{gov\_closeness\_w\_01}}
\newcommand{\techno}{\texttt{Technocracy}}
\newcommand{\demsat}{\texttt{Q8\_1}}      % democratic satisfaction
\newcommand{\trustpol}{\texttt{Q9\_4}}    % trust in Finnish politicians
\newcommand{\ind}[1]{\mathbbm{1}\!\left\{#1\right\}}

%\title{Losers, Delegation, and the Ideological Valence of Expertise: Evidence from Finland}
\vspace{-1cm}\title{\textbf{\input{title.txt}\unskip}} % Article title





\author[1]{\textsc{Katri Aaltonen}\thanks{\href{mailto:katri.m.aaltonen@utu.fi}{katri.m.aaltonen@utu.fi}; \href{https://www.utu.fi/fi/ihmiset/katri-aaltonen}{\texttt{https://www.utu.fi/fi/ihmiset/katri-aaltonen}}}}
\author[2]{\textsc{Hector Bahamonde}\thanks{\href{mailto:hector.bahamonde@utu.fi}{hector.bahamonde@utu.fi}; \href{https://www.hectorbahamonde.com}{\texttt{https://www.hectorbahamonde.com}}}}
\author[3]{\textsc{Mikko Niemelä}\thanks{\href{mailto:mikko.niemela@utu.fi}{mikko.niemela@utu.fi}; \href{https://www.utu.fi/fi/ihmiset/mikko-niemela}{\texttt{https://www.utu.fi/fi/ihmiset/mikko-niemela}}. \\{\bf {\color{blue}Authors are listed in alphabetical order; all authors contributed equally}.}}}


\affil[1]{Academy Research Fellow, INVEST Research Flagship Centre, University of Turku, Finland}
\affil[2]{Senior Researcher, INVEST Research Flagship Centre, University of Turku, Finland}
\affil[3]{Professor of Sociology, INVEST Research Flagship Centre, University of Turku, Finland}

%\affil[ ]{\textit{Authors are listed in alphabetical order; all authors contributed equally.}}




\date{\today}
\IfFileExists{upquote.sty}{\usepackage{upquote}}{}
\begin{document}


\maketitle
\thispagestyle{empty}
















%\newpage
\vspace{-1cm}
\begin{abstract}
\input{abstract.txt}\unskip
\end{abstract}



\vspace*{0.3cm}
\centerline{{\bf Abstract length}: 276 words.}
\vspace*{0.3cm}



%\centerline{\bf Please consider downloading the last version of the paper \href{https://raw.githubusercontent.com/hbahamonde/democratic_backsliding/main/2025/Dem_Backsliding_2.pdf}{\texttt{{\color{red}here}}}.}

\vspace*{0.3cm}
\centerline{\bf {\color{red}PLEASE DO NOT CIRCULATE}.}

\vspace*{0.5cm}
\centerline{\providecommand{\keywords}[1]{\textbf{\textit{Keywords---}} #1} % keywords.  
\keywords{{\input{keywords.txt}\unskip}}}



\clearpage
\pagenumbering{arabic}
\setcounter{page}{1}



\section{Framing, Delegation, and Mass Preferences}

Healthcare systems in affluent democracies face a striking puzzle. Authority over high-cost and controversial treatment decisions is frequently delegated to insulate policy from political pressure—yet mass opinion over these decisions remains highly elastic to how choices are presented. This elasticity raises an important question: Does institutional delegation insulate these decisions from framing effects, or do citizens' preferences continue to shift with how decisions are framed even when authority is formally insulated from electoral politics? This article shows that mass preferences remain elastic to framing effects: preferences over healthcare funding decisions are reference-point dependent regardless of institutional insulation. Even when clinical evidence, costs, and uncertainty are held constant, citizens' support for expensive medicines changes substantially depending on whether the decision is presented as avoiding a salient loss---withholding a last-resort option from an identifiable patient---or as preserving existing gains by protecting finite fiscal resources from extraordinary expenditure. We find that delegated decision-making does not eliminate the contestability of distributive choices; rather, it reorganizes contestation around public communication and the interpretive frames through which citizens evaluate state action. Our findings have important implications: such healthcare policies remain politically contentious not because decisions are made by elected politicians rather than experts, but because mass preferences remain anchored to loss-versus-gain framing and to rescue narratives that can be activated (or dampened) through how choices are presented to the public.

The material stakes of these choices are substantial and structurally enduring. \textcite{Torkki2022} find that medicine costs increased rapidly across Nordic countries between 2012 and 2017, with growth rates ranging from 37 to 125 percent, driven in large part by expensive oncology medicines with uncertain clinical impact. The same study documents pronounced cross-national differences in cost structures and cost drivers in cancer care, reflecting variation in the organization of care and health policy priorities \parencite{Torkki2022}. Similar dynamics extend beyond Europe. Using U.S. data, \textcite{Nguyen2025} report that very high-cost users tend to live in areas with higher social needs and that conditions such as cancer account for a disproportionate share of expenditures, underscoring the distributive and political dimensions of high-cost care. These pressures are compounded by scientific uncertainty: novel oncology drugs often enter reimbursement debates with limited evidence on long-term survival or quality-of-life effects, while the fiscal implications of public funding are immediate. As \textcite{Aziz2020} illustrate, marginal expected health gains are frequently paired with exceptionally high costs.

This context generates a canonical tension between two logics that structure how citizens evaluate state funding decisions, and it provides the substantive terrain on which framing mechanisms operate.\footnote{At a structural level, these choices resemble classic problems in distributive politics, where public funds are allocated to projects with concentrated benefits and diffuse costs, such as geographically targeted infrastructure spending \parencite{Weingast1981a}. However, healthcare rationing differs in a crucial respect: while residence in a remote island is largely a matter of choice, exposure to severe illness is fundamentally uncertain and broadly shared. As a result, citizens evaluating cancer medicines may support high-cost spending not because they benefit directly today, but because they recognize a latent personal risk of future need—an intertemporal logic that distinguishes healthcare from standard pork-barrel allocations.} One logic emphasizes stewardship: because healthcare budgets are finite, publicly financed systems require allocating resources in ways that sustain broad access and protect population health. The other logic emphasizes rescue: when a specific patient faces imminent death and a treatment offers even a small chance of benefit, refusing to act is experienced as a morally troubling choice. This intuition is captured in the ``rule of rescue,'' which \textcite[2407]{McKie2003} define as the imperative people feel to rescue identifiable individuals facing avoidable death. ``This imperative entails a preference for identifiable over statistical lives, the shock-horror response it elicits, [and] the preference it entails for lifesaving over non-lifesaving measures.'' In cancer care, these intuitions are repeatedly activated because patients are identifiable, prognoses are often stark, and clinical narratives translate easily into public dramas of avoidable loss. At the same time, these decisions are unavoidably distributive: the opportunity costs of funding one medicine are borne by anonymous others, and citizens must assess whether collective resources have been allocated appropriately.

Existing research indicates that citizens can engage with these trade-offs, yet their expressed preferences are conditional on how decisions are presented and justified. Deliberative exercises suggest that publics can accept scarcity and the need for thresholds under certain procedural conditions. In British Columbia, \textcite{Bentley2018} find that participants accepted the principle of resource scarcity and the need for governments to make difficult trade-offs when allocating healthcare resources, and supported the view that cost-benefit thresholds must be set for high-cost drugs. Participants also expected reasonable health benefits in return for large expenditures and supported the view that some drugs do not merit funding. Survey evidence from other contexts reveals similar conditionality. For example, in South Korea, \textcite{Noh2025} report that three-quarters of respondents supported reimbursement of high-cost cancer drugs, but that support depended on confidence that drugs were safe, effective, and cost-effective.

However, mass political attitudes under conditions of constraint are highly elastic to how choices are presented. A broad literature on framing demonstrates that preferences over public spending are malleable. In Ghana, \textcite{Smith2025a} show that support for health taxes increased when these taxes were framed as improving public health or promoting fairness. Similarly, \textcite{Jones2022a} argue that healthcare policy evaluation necessarily depends on which considerations are made salient. Importantly, others show that framing alters how risks and outcomes are interpreted. For example, \textcite{Kahneman1984a} show that preferences vary markedly when identical outcomes are framed in terms of losses rather than gains. Related work shows that frames shape mass policy attitudes in other domains, including food regulation \parencite{Roh2016} and medical debt \parencite{McCabe2023}. During COVID-19, \textcite{Valenzuela2021} document how competing frames shaped support for restrictive policies, with exposure to both economic and public health frames reducing support for mobility restrictions \parencite{Valenzuela2021}.

Taken together, these literatures motivate an interpretation of mass preferences in delegated healthcare rationing that emphasizes reference points, perceived losses, and asymmetric responses to risk. Prospect theory establishes that individuals evaluate choices relative to reference points and exhibit loss aversion, becoming risk-seeking when choices are framed in the loss domain and risk-averse when framed in the gain domain \parencite{Tversky1985,Vis2011a}. Applied to healthcare rationing, this predicts that when a contested funding choice is framed as avoiding a salient loss---for example, by emphasizing withholding a last-resort option from an identifiable patient---prospect theory predicts citizens should be more willing to endorse riskier and more expensive options to avert that loss \parencite{Hameleers2021}. Conversely, when the same choice is framed as protecting existing gains---safeguarding finite fiscal resources from extraordinary expenditure---citizens should favor restraint \parencite{Hameleers2021}. The key question is whether these framing effects persist when such decisions are delegated to non-majoritarian authorities insulated from electoral accountability. Delegation is commonly narrated as a route to depoliticization, in which evidence is assessed through objective criteria, thresholds are applied consistently, and distributive choices are presented as technically grounded rather than politically discretionary \parencite{Thatcher2002}. Yet if mass preferences are systematically sensitive to loss-versus-gain frames \parencite{Laenen2025}, then institutional insulation from electoral politics may not eliminate contestation. Instead, contestation may be reorganized around how decisions are communicated and the interpretive frames through which citizens evaluate state action---shifting from political contestation to contestation over meaning and narrative authority.


This implication connects to debates on delegation and accountability, where responsibility for policy outcomes becomes a central object of political contestation. Existing work shows that delegation is often assumed to allow governments to manage electoral pressures by shifting responsibility for unpopular decisions, but that this effect is conditional. For example, \textcite{Heinkelmann-Wild2023} show that delegation can facilitate blame avoidance only under specific institutional designs, while \textcite{Heinkelmann-Wild2020} demonstrate that the redirection of political pressure depends on whether citizens believe governments can credibly deny responsibility for outcomes. Delegation can therefore redistribute political pressure, but it does not eliminate contestation. In parallel, research on the contestability of expertise highlights that public responses to delegated decisions cannot be explained by technical merit alone. As \textcite{Head2024} argues, expert policy advice is increasingly challenged by diverse actors, making the interpretation of state action central.


Another case in point is deliberative democracy, which is often proposed as a response to contested distributive decisions. While deliberative approaches promise that transparent, reason-giving procedures can facilitate acceptance of difficult choices, critics argue that consensus-seeking can depoliticize conflict rather than resolve it. \textcite{Pellizzoni2001} argue that deliberation obscures contexts characterized by deep value conflict and radical uncertainty. \textcite{Swyngedouw2009} similarly contend that technocratic governance framed around consensus can foreclose political disagreement and relocate conflict rather than settle it. In healthcare rationing, deliberation and delegation may thus shift contestation away from representative politics while leaving mass evaluations to be shaped primarily through framing. In fact, recent work systematizes these concerns. \textcite{Schafer2023} review critiques of deliberative democracy that emphasize its tendency to reproduce conservative outcomes under adverse conditions, while \textcite{Gaus2020} argue that deliberative venues can entrench expert dominance. Related critiques suggest that consensus-led governance does not resolve conflict so much as reconfigure it, masking adversarial relations \parencite{Fougere2020} and channeling contestation into managed forms of participation \parencite{Duncan2018}.

These debates point to a clear research gap. While a large literature documents framing effects on preferences under risk, and another examines delegation, blame, and accountability, these strands have rarely been integrated to assess whether \emph{framing shapes mass preferences over state funding decisions when authority is delegated}. Existing work suggests that citizens’ responses to public policy depend on expectations, trust, and interpretive cues rather than institutional design alone \parencite{Laenen2025}, and that accountability in the regulatory state operates through reputational judgments formed across audiences \parencite{GomezDiaz2025}. What remains unknown is whether delegation constrains the elasticity of mass preferences to framing; in short, whether reference-point–dependent evaluations continue to structure public support for costly and uncertain distributive choices even when decision-making is formally insulated from electoral politics.


This article contributes by identifying and interpreting experimentally induced framing effects on mass preferences over state funding decisions in a domain characterized by scarcity, uncertainty, and salient distributive trade-offs. We argue that framing does more than shift stated risk preferences in the abstract: in delegated healthcare rationing, it changes the reference point through which citizens evaluate whether the state should fund an expensive and uncertain medicine. When decisions are framed as loss avoidance and rescue, citizens are more likely to support funding and to assign responsibility for inaction. When decisions are framed as protecting gains and safeguarding finite collective resources, citizens are more likely to endorse restraint and cost containment. Using a population-based survey experiment embedded in the 2021 Finnish Medicines Barometer, where respondents evaluate an identical clinical vignette under alternative frames, the study clarifies how prospect theory helps explain why mass preferences over the same funding choice can shift with communication, and why such shifts are politically consequential in policy domains where delegation is intended to insulate decisions from politics but instead reorganizes how mass preferences are formed and expressed.

{\color{red}This paper proceeds as follows. \autoref{theory} explains our theory, focusing in how prospect theory and the rule of rescue....}



\section{Prospect Theory and the Rule of Rescue in Healthcare Rationing}\label{theory}

Mass preferences over healthcare rationing are structured by two complementary logics: reference-dependent evaluation under risk, as formalized by prospect theory, and the moral psychology of rescue, which captures the disproportionate weight citizens assign to identifiable individuals facing avoidable death. Together, these frameworks explain why framing effects persist even when distributive decisions are delegated to expert bodies insulated from electoral politics. Delegation alters who formally decides, but it does not alter how citizens cognitively and emotionally evaluate outcomes. As a result, preferences over healthcare funding depend critically on whether choices are framed as preventing losses to identifiable individuals or as protecting existing collective gains, and this dependence remains politically consequential under institutional insulation.

Prospect theory departs from expected utility by demonstrating that individuals evaluate outcomes relative to reference points rather than final states \textcite[263]{Kahneman1979}. The value function is defined over gains and losses relative to those reference points; it is concave for gains, convex for losses, and steeper in the loss domain, implying loss aversion \textcite[342]{Kahneman1984}. Extensive evidence confirms that reference dependence and loss aversion are among the most robust empirical regularities in decision-making \textcite{Barberis2013}. Reference points themselves are endogenous. They are shaped by expectations, prior experience, and salient information \textcite{Koszegi2006,Koszegi2007}. In uncertain environments such as healthcare rationing, where clinical benefits are probabilistic and fiscal constraints are salient, gain-loss utility systematically influences evaluation. Whether a denied treatment is perceived as a neutral cost or as an uncompensated loss depends on how expectations are constructed and communicated \textcite[205]{Kahneman1984,Kahneman2003,Kumar2024}.

Loss aversion generates the reflection effect: individuals tend to be risk-averse when outcomes are framed as gains and risk-seeking when identical outcomes are framed as losses \textcite[273]{Kahneman1979}. This pattern has been replicated across stakes and populations, including high-incentive laboratory settings and real-world market behavior \textcite{Holt2002,List2004,Knetsch1989}. Extensions of prospect theory incorporating cumulative outcomes and rank-dependent decision weights further demonstrate that loss aversion persists in complex, multi-stage environments \textcite{Schmidt2008}. In healthcare contexts, this implies that citizens may endorse risky and expensive treatments when refusal is framed as a loss, even when expected benefits are modest.

Prospect theory also incorporates probability weighting: individuals overweight small probabilities and underweight large ones, producing nonlinear responses to risk \textcite[282–285]{Kahneman1979}. Combined with loss aversion, this yields the fourfold pattern of risk attitudes \textcite{Kahneman1984}. This pattern is particularly relevant for novel oncology medicines, which involve low probabilities of substantial benefit and high stakes. Evidence shows that even trained professionals and political elites display probability distortions consistent with prospect theory \textcite{Abdellaoui2013,Linde2017,Roisse2022}. Thus, uncertainty does not neutralize framing effects; instead, it amplifies them.

Reference dependence also manifests as status quo bias and the endowment effect. Individuals value existing entitlements more than equivalent foregone alternatives because relinquishing them is experienced as a loss \textcite{Kahneman1991,Knetsch1989,Kahneman1990,Thaler1980}. Once expectations of access to treatment are established, withholding care is evaluated as loss rather than allocation. Evidence from insurance and risk management further shows that protection against losses looms larger than equivalent gains \textcite{Hwang2021,Pressacco1984,Meier1999,Tsang2020}. These mechanisms are directly relevant to healthcare rationing, where expectations of coverage anchor reference points.

Framing effects arise when identical choice problems activate different reference points. Classic demonstrations show systematic preference reversals when outcomes are framed as gains versus losses \textcite{Tversky1981,Kahneman1984}. Psychological research identifies multiple mechanisms through which framing operates: cognitive accessibility, emotional activation, and pragmatic inference \textcite{Flusberg2024}. Loss frames reliably elicit stronger affective responses and increase willingness to accept risk \textcite{Hameleers2021}. Heuristics such as availability and representativeness further reinforce framing effects by making certain outcomes more salient \textcite{Tversky1973,Tversky1974,Lau2001}. Importantly, these mechanisms operate independently of institutional context; they are features of human cognition.

Empirical work across policy domains demonstrates the generality of framing effects. Preferences over healthcare, taxation, labor market regulation, environmental policy, and social spending all vary with how choices are labeled and narrated \textcite{Holl2018,Smith2025a,Schaitberger2025,Roh2016,DeGolia2019,Svenningsen2021,Asticher2025}. During crises such as COVID-19, competing frames reshaped support for restrictive policies \textcite{Valenzuela2021}. These findings indicate that framing effects are not domain-specific anomalies but manifestations of stable cognitive processes.

The rule of rescue complements prospect theory by explaining why certain losses are especially salient. It captures the widely observed imperative to save identifiable individuals facing avoidable death, even at high cost \textcite[2407]{McKie2003}. Rescue narratives privilege identifiable over statistical lives and prioritize lifesaving interventions over efficiency. Extensions emphasize the narrative dimension of rescue: preserving opportunities for closure and meaning at the end of life \textcite{Sinclair2022}. In healthcare rationing, identifiable cancer patients readily activate rescue intuitions, transforming funding refusals into morally charged losses.

Political science applications of prospect theory demonstrate that reference-dependent risk-taking extends to collective decision-making. Loss aversion helps explain escalation in international conflict, electoral risk-taking, welfare reform, and policy change \textcite{Levy1992,Levy1997,McDermott1998,Vis2007,Vis2011,Weyland2002,Carreras2019}. Framing by elites shapes mass preferences in distributive politics \textcite{Marx2014}. These studies establish that prospect theory is not confined to individual choice but applies to politically consequential judgments.

Crucially, delegation does not neutralize these mechanisms. While delegation is often justified as a route to depoliticization, evidence shows that it redistributes rather than eliminates contestation \textcite{Heinkelmann-Wild2023,Heinkelmann-Wild2020}. Accountability under delegation operates through interpretation and reputation rather than formal responsibility \textcite{Head2024}. Deliberative and technocratic arrangements similarly risk relocating conflict without resolving it \textcite{Pellizzoni2001,Swyngedouw2009,Fougere2020,Duncan2018}. From a cognitive perspective, delegation changes institutional distance but not reference-point formation.

Integrating prospect theory and the rule of rescue yields clear expectations for delegated healthcare rationing. First, preferences should reverse when identical funding decisions are framed as loss avoidance versus gain protection. Second, identifiability should magnify loss framing by activating rescue intuitions. Third, reference points should be shaped by communication and narrative rather than by formal cost-effectiveness criteria. Fourth, these effects should persist under delegation because they arise from cognitive and moral processes rather than institutional design \textcite{Flusberg2024}. Empirically, this framework explains why mass preferences over funding an identical medicine shift with framing, as observed in the Finnish Medicines Barometer experiment.

The broader implication is that expertise and institutional insulation operate within, not outside, a framing-sensitive evaluative environment. Citizens may endorse expert decision-making in principle, yet respond elastically when specific cases are framed as denials of rescue or as threats to collective stewardship. Such responses reflect neither misunderstanding nor instability, but the predictable operation of loss aversion and rescue morality. Delegation reorganizes the arena of contestation, but framing determines how mass preferences are formed and expressed.


\section{Empirical Section}

The empirical analysis draws on data from the 2021 Finnish Medicines Barometer, a national, cross-sectional population survey administered biennially by the Finnish Medicines Agency to examine experiences, opinions, and values related to health, medicines, and well-being. The 2021 wave included an ad hoc experimental module designed specifically to assess public attitudes toward the public funding of novel, high-cost oncology medicines characterized by uncertain clinical benefit. Finland provides a particularly appropriate research context for examining these questions. The Finnish healthcare system is comprehensive and publicly funded, closely resembling other advanced welfare states in which questions of healthcare rationing and legitimacy are politically salient. Moreover, Nordic healthcare systems face precisely the pressures that motivate this study. As \textcite[1216--1222]{Torkki2022} document, ``cancer care in Nordic countries has significant differences in both cost structures and in the development of cost drivers,'' driven in substantial part by expensive pharmaceutical innovations with uncertain value.

Data collection was carried out by a professional market research company (Taloustutkimus Ltd) using a pre-recruited online panel of approximately 40,000 Finnish citizens. To reach the target sample size, 10,105 invitations were distributed to panel members. The sample was stratified to ensure balance across gender, age, education level, and geographic region, with eligibility restricted to Finnish-speaking citizens aged 18--79 years. The final analytic sample consists of 2,081 respondents, corresponding to a completion rate of approximately 20.6 percent. While online panels are not probability samples, stratification and quota-based recruitment ensure close correspondence with the Finnish adult population on key sociodemographic dimensions, making the data suitable for inference about mass public opinion in a welfare-state context.

The experimental module employed a randomized between-subjects design focused on framing effects. Although the module included both an information experiment and a framing experiment, the analysis presented here focuses exclusively on the framing experiment. Respondents were randomly assigned to one of three experimental conditions—Scenarios B, C, and D—each administered to approximately 500 respondents. All respondents received an identical clinical case description and decision-making task; the experimental manipulation consisted solely of the framing information presented between the case description and the decision task. This design ensures that any observed differences in preferences across conditions can be attributed to framing rather than to differences in substantive information.

The clinical case described a new medicine indicated for the treatment of a specific incurable cancer. The description emphasized several features that mirror real-world oncology policy decisions. Respondents were informed that the medicine does not cure the disease, that laboratory test results suggest it destroys cancer cells in approximately one-third of patients who receive it, and that it remains unknown whether the medicine extends patients’ lives or improves quality of life relative to existing treatment options. The case further specified that the medicine carries a substantial risk of adverse effects and would be prescribed to a very small patient population—approximately 10 to 20 patients annually in Finland. Importantly, respondents were told that if approved for public funding, the medicine would cost more than 60,000 euros per patient at the final stage of cancer treatment. This cost level reflects the magnitude of expenditures associated with innovative cancer medicines and is consistent with empirical estimates reported in the health economics literature (see \textcite[1]{Aziz2020}). By combining high cost, uncertain benefit, and limited patient eligibility, the case captures the core elements of contemporary cancer drug reimbursement dilemmas.

Following the case description, respondents in the control condition (Scenario B) proceeded directly to the decision-making task without additional framing, receiving only the clinical and cost information. In contrast, respondents in the two treatment conditions received additional framing statements designed to shift the reference point through which the decision was evaluated. In the loss-framed condition (Scenario C), respondents were told: ``There is no cure for this particular type of cancer. The new medicine is a possible option for patients who have already received multiple treatments and for whom the remaining options are limited.'' This framing emphasizes the absence of alternatives and the patient’s proximity to death, thereby activating a loss-domain reference point centered on the avoidable loss of life. In the gains-framed condition (Scenario D), respondents were told: ``The funds available for healthcare are finite. The adoption of the new medicine means that the funds used to pay for it will not be available elsewhere in healthcare.'' This framing highlights opportunity costs and system-level trade-offs, activating a gains-domain reference point focused on protecting existing healthcare provision. Both frames reflect language commonly used in real policy debates and are theoretically motivated by prospect theory’s emphasis on reference-point-dependent risk preferences.

After exposure to the case description and, where applicable, the framing manipulation, all respondents completed the same decision-making task. They were asked: ``What kind of decision regarding the new medicine’s use would you find acceptable? Please choose the option that best reflects your opinion.'' Four response options were provided: unconditional public funding regardless of price; conditional funding contingent on price reductions; rejection of public funding; and an explicit ``I don’t know'' option. These response categories capture meaningful variation in attitudes toward risk and cost in healthcare spending. Unconditional funding represents the most risk-seeking option, accepting both high financial cost and uncertain clinical benefit. Conditional funding reflects moderate risk aversion, accepting treatment only if financial exposure is reduced. Rejection represents the most risk-averse position, prioritizing resource protection over uncertain benefit. The inclusion of an explicit uncertainty option allows respondents to express ambivalence rather than forcing artificial choice.

Prospect theory yields clear expectations for how preferences should vary across framing conditions. When the decision context is framed as loss avoidance—emphasizing the imminent death of identifiable patients and the absence of alternatives—citizens are expected to become more risk-seeking, increasing support for unconditional funding and reducing outright rejection. When the same decision is framed in terms of protecting gains—emphasizing finite budgets and opportunity costs—citizens are expected to exhibit greater risk aversion, shifting support toward conditional funding or rejection. Because the clinical information and decision options are held constant across conditions, observed differences in preferences can be interpreted as evidence of framing-induced shifts in reference points and risk orientation rather than changes in substantive beliefs about the medicine itself.



\begin{knitrout}
\definecolor{shadecolor}{rgb}{0.969, 0.969, 0.969}\color{fgcolor}\begin{figure}

{\centering \includegraphics[width=\maxwidth]{figure/pred-1} 

}

\caption[Framing Effects on Public Support for Funding a High-Cost Cancer Medicine]{Framing Effects on Public Support for Funding a High-Cost Cancer Medicine}\label{fig:pred}
\end{figure}

\end{knitrout}






\FloatBarrier
\newpage



\clearpage
\newpage
\pagenumbering{roman}
\setcounter{page}{1}
\printbibliography
\clearpage
\newpage



%%%%%%%%%%%%%%%%%%%%%%%%%%%%%%%%%%%%%%%%%%%%%%
% WORD COUNT
%%%%%%%%%%%%%%%%%%%%%%%%%%%%%%%%%%%%%%%%%%%%%%
\clearpage



\begin{center}
\vspace*{\stretch{1}}
\dotfill
\dotfill {\huge {\bf Word count}: 4,291} \dotfill
\dotfill
\vspace*{\stretch{1}}
\end{center}

\clearpage

%%%%%%%%%%%%%%%%%%%%%%%%%%%%%%%%%%%%%%%%%%%%%%
% WORD COUNT
%%%%%%%%%%%%%%%%%%%%%%%%%%%%%%%%%%%%%%%%%%%%%%


% Online Appendix
%\newpage
%\section{Online Appendix}
%\pagenumbering{Roman}
%\setcounter{page}{1}



%% reset tables and figures counter
\setcounter{table}{0}
\renewcommand{\thetable}{A\arabic{table}}
\setcounter{figure}{0}
\renewcommand{\thefigure}{A\arabic{figure}}



\section{Appendix}
\pagenumbering{roman}
\setcounter{page}{1}


\subsection{Regression Models}
\FloatBarrier

\begin{table}[!htbp]
\centering
\caption{Framing Effects on Public Support for Funding Expensive Cancer Medicines}
\label{tab:ologit-models}
%\setlength{\tabcolsep}{3.5pt} % tighten intercolumn spacing
\begin{tabular}{ll}
\hline
& Ordinal logit \\ \hline
Rule of rescue (vs. control) & \num{1.478}*** \\
& [\num{1.252}, \num{1.746}] \\
Utility maximizing (vs. control) & \num{0.762}** \\
& [\num{0.644}, \num{0.901}] \\
Male (vs. female) & \num{0.953} \\
& [\num{0.830}, \num{1.095}] \\
Age & \num{1.016}*** \\
& [\num{1.012}, \num{1.021}] \\
Income: middle (vs. low) & \num{0.858} \\
& [\num{0.719}, \num{1.023}] \\
Income: high (vs. low) & \num{0.944} \\
& [\num{0.778}, \num{1.147}] \\
Income: other/unknown & \num{0.821} \\
& [\num{0.652}, \num{1.035}] \\
Eligible for Kela reimbursement & \num{0.977} \\
& [\num{0.829}, \num{1.151}] \\
Num.Obs. & \num{2460} \\
\hline
\end{tabular}

\par\vspace{0.4ex}
\footnotesize \emph{Notes:} Entries report odds ratios from an ordinal logistic regression. The dependent variable measures respondents' preferred funding decision for a novel, high-cost cancer medicine, ordered from unconditional public funding to outright rejection. The key independent variable captures experimental framing of the decision context. Control variables include gender, age, income group, education, eligibility for Kela medicine reimbursement, and self-reported medicine expenditure. Models are estimated using survey weights. 90\% confidence intervals in parentheses. $^{*}p<0.10$, $^{**}p<0.05$, $^{***}p<0.01$.
\end{table}



\end{document}





