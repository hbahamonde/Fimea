% !Rnw weave = knitr
% !TeX program = pdflatex

%========================
% paper.tex
%========================
\documentclass[11pt]{article}\usepackage[]{graphicx}\usepackage[]{xcolor}
% maxwidth is the original width if it is less than linewidth
% otherwise use linewidth (to make sure the graphics do not exceed the margin)
\makeatletter
\def\maxwidth{ %
  \ifdim\Gin@nat@width>\linewidth
    \linewidth
  \else
    \Gin@nat@width
  \fi
}
\makeatother

\definecolor{fgcolor}{rgb}{0.345, 0.345, 0.345}
\newcommand{\hlnum}[1]{\textcolor[rgb]{0.686,0.059,0.569}{#1}}%
\newcommand{\hlsng}[1]{\textcolor[rgb]{0.192,0.494,0.8}{#1}}%
\newcommand{\hlcom}[1]{\textcolor[rgb]{0.678,0.584,0.686}{\textit{#1}}}%
\newcommand{\hlopt}[1]{\textcolor[rgb]{0,0,0}{#1}}%
\newcommand{\hldef}[1]{\textcolor[rgb]{0.345,0.345,0.345}{#1}}%
\newcommand{\hlkwa}[1]{\textcolor[rgb]{0.161,0.373,0.58}{\textbf{#1}}}%
\newcommand{\hlkwb}[1]{\textcolor[rgb]{0.69,0.353,0.396}{#1}}%
\newcommand{\hlkwc}[1]{\textcolor[rgb]{0.333,0.667,0.333}{#1}}%
\newcommand{\hlkwd}[1]{\textcolor[rgb]{0.737,0.353,0.396}{\textbf{#1}}}%
\let\hlipl\hlkwb

\usepackage{framed}
\makeatletter
\newenvironment{kframe}{%
 \def\at@end@of@kframe{}%
 \ifinner\ifhmode%
  \def\at@end@of@kframe{\end{minipage}}%
  \begin{minipage}{\columnwidth}%
 \fi\fi%
 \def\FrameCommand##1{\hskip\@totalleftmargin \hskip-\fboxsep
 \colorbox{shadecolor}{##1}\hskip-\fboxsep
     % There is no \\@totalrightmargin, so:
     \hskip-\linewidth \hskip-\@totalleftmargin \hskip\columnwidth}%
 \MakeFramed {\advance\hsize-\width
   \@totalleftmargin\z@ \linewidth\hsize
   \@setminipage}}%
 {\par\unskip\endMakeFramed%
 \at@end@of@kframe}
\makeatother

\definecolor{shadecolor}{rgb}{.97, .97, .97}
\definecolor{messagecolor}{rgb}{0, 0, 0}
\definecolor{warningcolor}{rgb}{1, 0, 1}
\definecolor{errorcolor}{rgb}{1, 0, 0}
\newenvironment{knitrout}{}{} % an empty environment to be redefined in TeX

\usepackage{alltt}

%---- Page & fonts
\usepackage[margin=1in]{geometry}
\usepackage[T1]{fontenc}
\usepackage[utf8]{inputenc}
\DeclareUnicodeCharacter{2013}{--}   % en dash
\DeclareUnicodeCharacter{2014}{---}  % em dash
\DeclareUnicodeCharacter{2212}{-}    % unicode minus
\usepackage{lmodern}
\usepackage[american]{babel}
\usepackage{microtype}
\usepackage{authblk}

%---- Math & symbols
\usepackage{amsmath, amssymb, mathtools}
\usepackage{bbm}          % for \mathbbm{1} indicator
\usepackage{siunitx}
\usepackage{dcolumn}   % needed for stargazer's D{.}{.}{-3} columns
\usepackage{csquotes}  % recommended with biblatex (silences that warning)

%---- Graphics, tables, floats
\usepackage{graphicx}
\graphicspath{{fig/}{figures/}{build/}} % adjust as needed
\usepackage{booktabs, threeparttable}
\usepackage{caption}
\usepackage{subcaption}
\usepackage{float}
\usepackage{placeins}

\usepackage{tabularray}
\usepackage{codehigh}
\usepackage[normalem]{ulem}
\UseTblrLibrary{booktabs}
\UseTblrLibrary{siunitx}
\newcommand{\tinytableTabularrayUnderline}[1]{\underline{#1}}
\newcommand{\tinytableTabularrayStrikeout}[1]{\sout{#1}}
\NewTableCommand{\tinytableDefineColor}[3]{\definecolor{#1}{#2}{#3}}


%---- % Allows abstract customization
\usepackage{abstract} 
\renewcommand{\abstractnamefont}{\normalfont\bfseries} % Set the "Abstract" text to bold
%\renewcommand{\abstracttextfont}{\normalfont\small\itshape} % Set the abstract itself to small italic text



%---- Links & clever refs
\usepackage[hidelinks]{hyperref}
\usepackage[capitalize,noabbrev]{cleveref}

%---- Citations & bib
\usepackage[backend=biber,style=authoryear,dashed=false,doi=false,isbn=false,url=false,arxiv=false]{biblatex}
\addbibresource{/Users/hectorbahamonde/Bibliografia_PoliSci/library.bib}  % your .bib file


%---- Custom macros (edit as you like)
\newcommand{\govdist}{\texttt{Government distance}}
\newcommand{\govclose}{\texttt{gov\_closeness\_w\_01}}
\newcommand{\techno}{\texttt{Technocracy}}
\newcommand{\demsat}{\texttt{Q8\_1}}      % democratic satisfaction
\newcommand{\trustpol}{\texttt{Q9\_4}}    % trust in Finnish politicians
\newcommand{\ind}[1]{\mathbbm{1}\!\left\{#1\right\}}

%\title{Losers, Delegation, and the Ideological Valence of Expertise: Evidence from Finland}
\vspace{-1cm}\title{\textbf{\input{title.txt}\unskip}} % Article title





\author[1]{\textsc{Katri Aaltonen}\thanks{\href{mailto:katri.m.aaltonen@utu.fi}{katri.m.aaltonen@utu.fi}; \href{https://www.utu.fi/fi/ihmiset/katri-aaltonen}{\texttt{https://www.utu.fi/fi/ihmiset/katri-aaltonen}}}}
\author[2]{\textsc{Hector Bahamonde}\thanks{\href{mailto:hector.bahamonde@utu.fi}{hector.bahamonde@utu.fi}; \href{https://www.hectorbahamonde.com}{\texttt{https://www.hectorbahamonde.com}}}}
\author[3]{\textsc{Mikko Niemelä}\thanks{\href{mailto:mikko.niemela@utu.fi}{mikko.niemela@utu.fi}; \href{https://www.utu.fi/fi/ihmiset/mikko-niemela}{\texttt{https://www.utu.fi/fi/ihmiset/mikko-niemela}}. \\{\bf {\color{blue}Authors are listed in alphabetical order; all authors contributed equally}.}}}


\affil[1]{Academy Research Fellow, INVEST Research Flagship Centre, University of Turku, Finland}
\affil[2]{Senior Researcher, INVEST Research Flagship Centre, University of Turku, Finland}
\affil[3]{Professor of Sociology, INVEST Research Flagship Centre, University of Turku, Finland}

%\affil[ ]{\textit{Authors are listed in alphabetical order; all authors contributed equally.}}




\date{\today}
\IfFileExists{upquote.sty}{\usepackage{upquote}}{}
\begin{document}


\maketitle
\thispagestyle{empty}
















%\newpage
\vspace{-1cm}
\begin{abstract}
\input{abstract.txt}\unskip
\end{abstract}



\vspace*{0.3cm}
\centerline{{\bf Abstract length}: 2 words.}
\vspace*{0.3cm}



%\centerline{\bf Please consider downloading the last version of the paper \href{https://raw.githubusercontent.com/hbahamonde/democratic_backsliding/main/2025/Dem_Backsliding_2.pdf}{\texttt{{\color{red}here}}}.}

\vspace*{0.3cm}
\centerline{\bf {\color{red}PLEASE DO NOT CIRCULATE}.}

\vspace*{0.5cm}
\centerline{\providecommand{\keywords}[1]{\textbf{\textit{Keywords---}} #1} % keywords.  
\keywords{{\input{keywords.txt}\unskip}}}



\clearpage
\pagenumbering{arabic}
\setcounter{page}{1}



\section{Motivation}

Healthcare systems in affluent democracies increasingly confront a stark and persistent paradox: authority over expensive and controversial healthcare decisions is delegated in order to insulate them from day-to-day political pressure, yet the outcomes of these decisions remain politically contested in ways that \emph{reconfigure} rather than \emph{eliminate} conflict. Delegation is commonly narrated as a route to depoliticization. Under this view, evidence is assessed, thresholds are applied, and technically grounded decisions are reached such that elected officials can accept the outcomes at arm’s length. In practice, however, politics does not disappear; it migrates. Contestation shifts away from legislative bargaining toward mass public evaluations of state responsibility and fairness when decisions involve life-and-death stakes. When citizens evaluate whether the state should fund a novel, high-cost cancer medicine with uncertain benefits, they do not simply weigh clinical evidence against budgetary constraints. Instead, they interpret the state’s action through \emph{frames} that assign meaning to loss, responsibility, and obligation: whether the state has failed to rescue an identifiable patient facing avoidable death, or whether it has acted as a responsible steward by protecting a collectively financed healthcare system from extraordinary expenditure. This is the political core of delegated healthcare decision-making: delegation does not depoliticize; it reconfigures contestation.

The material stakes of this reconfigured contestation are substantial and structurally enduring. \textcite[1216]{Torkki2022} show that medicine costs increased rapidly across Nordic countries between 2012 and 2017, with growth rates ranging from 37 to 125 percent, driven in large part by expensive oncology medicines with uncertain clinical impact. The same study documents pronounced cross-national differences in cost structures and cost drivers in cancer care, reflecting variation in the organization of care and health policy priorities \parencite[1216]{Torkki2022}. Similar dynamics extend beyond Europe. Using U.S. data, \textcite[2]{Nguyen2025} report that very high-cost users tend to live in areas with higher social needs and that conditions such as cancer account for a disproportionate share of expenditures, underscoring the distributive and political dimensions of high-cost care. These pressures are compounded by scientific uncertainty: novel oncology drugs often enter reimbursement debates with limited evidence on long-term survival or quality-of-life effects, while the fiscal implications of public funding are immediate. As \textcite[1]{Aziz2020} illustrate, marginal expected health gains are frequently paired with exceptionally high costs.

This context generates a canonical tension between two logics that structure how citizens evaluate state funding decisions. One logic emphasizes stewardship: because healthcare budgets are finite, publicly financed systems require allocating resources in ways that sustain broad access and protect population health. The other logic emphasizes rescue: when a specific patient faces imminent death and a treatment offers even a small chance of benefit, refusing to act is experienced as a morally troubling choice. This intuition is captured in the ``rule of rescue,'' which \textcite[2407]{McKie2003} define as the imperative people feel to rescue identifiable individuals facing avoidable death. ``This imperative entails a preference for identifiable over statistical lives, the shock-horror response it elicits, [and] the preference it entails for lifesaving over non-lifesaving measures.'' In cancer care, these intuitions are repeatedly activated because patients are identifiable, prognoses are often stark, and clinical narratives translate easily into public dramas of avoidable loss. At the same time, these decisions are unavoidably distributive: the opportunity costs of funding one medicine are borne by anonymous others, and citizens must assess whether collective resources have been allocated appropriately.

Existing research shows that citizens are capable of engaging with these trade-offs, yet often appear inconsistent across contexts. Deliberative exercises indicate that publics can accept scarcity and the need for thresholds under certain procedural conditions. In British Columbia, \textcite[1]{Bentley2018} find that participants accepted the principle of resource scarcity and the need for governments to make difficult trade-offs when allocating healthcare resources, and supported the view that cost-benefit thresholds must be set for high-cost drugs. Participants also expected reasonable health benefits in return for large expenditures and supported the view that some drugs do not merit funding. Importantly, these positions depended on how decisions were presented and justified. Survey evidence from other contexts reveals similar conditionality. In South Korea, \textcite[2–3]{Noh2025} report that three-quarters of respondents supported reimbursement of high-cost cancer drugs, but that support depended on confidence that drugs were safe, effective, and cost-effective. At the same time, respondents interpreted reimbursement decisions as reflecting both government responsibility and moral obligation \parencite[3]{Noh2025}.

However, mass political attitudes under conditions of constraint are far from stable. A broad literature on framing and justification shows that preferences over public spending are highly malleable. In Ghana, \textcite[831]{Smith2025a} demonstrate that support for health taxes increased when these taxes were framed as improving public health or promoting fairness. Similarly, \textcite[1]{Jones2022a} argue that healthcare policy evaluation necessarily involves multiple values and that public evaluations depend on which considerations are made salient. Behavioral research further demonstrates that framing alters how risks and outcomes are interpreted. \textcite[341]{Kahneman1984a} show that preferences over cancer treatments vary markedly when identical outcomes are framed in terms of mortality rather than survival. Related work shows that frames shape mass policy attitudes in other domains, including food regulation \parencite[9]{Roh2016} and medical debt \parencite[5]{McCabe2023}. During COVID-19, \textcite[1411–1430]{Valenzuela2021} document how competing frames shaped support for restrictive policies, with exposure to both economic and public health frames reducing support for mobility restrictions \parencite[1420]{Valenzuela2021}.

The central issue, however, is not only whether citizens support or oppose particular healthcare expenditures, but how mass preferences over state funding decisions become politically consequential under delegation. Delegation is often justified as a way for governments to manage distributive conflict by distancing themselves from unpopular decisions. Yet mass evaluations of responsibility do not follow institutional design mechanically; they are shaped by how outcomes are framed. \textcite[221]{Heinkelmann-Wild2023} show that delegation is often assumed to facilitate blame avoidance, but that such effects depend on delegation design. Similarly, \textcite[953–969]{Heinkelmann-Wild2020} demonstrate that blame shifting depends on whether citizens believe governments can credibly deny responsibility. Delegation can therefore redirect political pressure, but it does not eliminate contestation.

These dynamics are reinforced by the contestability of expertise. \textcite[156]{Head2024} describes how expert policy advice is increasingly challenged by diverse actors, making technical merit alone insufficient to secure public acceptance. What matters is how state action is interpreted. A refusal to fund a medicine can be framed as abandoning identifiable patients or as protecting a collectively financed system. These frames establish different evaluative standards through which citizens assess state decisions and assign responsibility.

Deliberative democracy is often proposed as a response to these challenges, but its limits are politically consequential. While deliberative approaches promise that transparent, reason-giving procedures can facilitate acceptance of difficult choices, critics argue that consensus-seeking can depoliticize conflict rather than resolve it. \textcite[59–86]{Pellizzoni2001} argue that deliberation obscures contexts characterized by deep value conflict and radical uncertainty. \textcite[601–620]{Swyngedouw2009} similarly contend that technocratic governance framed around consensus can foreclose political disagreement and relocate conflict rather than settle it. In healthcare rationing, deliberation and delegation may thus shift contestation away from representative politics while leaving mass evaluations to be shaped primarily through framing.

Recent work systematizes these concerns. \textcite[1–13]{Schafer2023} review critiques of deliberative democracy that emphasize its tendency to reproduce conservative outcomes under adverse conditions, while \textcite[335–347]{Gaus2020} argue that deliberative venues can entrench expert dominance. Related critiques show that consensus-led governance can paper over adversarial relations \parencite[683–699]{Fougere2020} and channel conflict into managed participation \parencite[1411–1424]{Duncan2018}. Together, this literature suggests that procedures alone cannot stabilize mass responses to distributive decisions.

These debates point to a core gap. While existing research examines delegation, blame, and accountability, and a separate literature documents framing effects on preferences under risk, these strands have not been integrated to examine whether framing shapes mass preferences over state funding decisions under delegation. In healthcare rationing, what is delegated is the state’s capacity to decide which medicines are publicly funded under conditions of scarcity and uncertainty, while citizens remain politically consequential as evaluators of those outcomes. Accountability in this context is not fixed but politically constructed. \textcite[159–169]{Zahariadis2020} show that accountability strategies depend on public expectations and trust, and \textcite[247–263]{Busuioc2016} argue that accountability in the regulatory state consists in managing and cultivating reputations vis-a-vis different audiences. Framing therefore provides the interpretive context through which citizens form preferences about contested state actions under delegation.

This framing-accountability nexus is intensified by strategic political incentives. \textcite[434–453]{Flinders2025} show that blame-seeking can be rational in polarized democracies. Anticipatory and reactive blame avoidance further shape state behavior \parencites[587–606]{Hinterleitner2017}[759–778]{Hinterleitner2023}. Appeals to expertise can also obscure responsibility; \textcite[466–485]{MacAulay2023} warn that ``follow the science’’ rhetoric can confuse the public about who chooses and who should be held accountable. In such contexts, framing becomes central to how mass publics interpret state action.

Concerns about reputation-based authority reinforce this point. \textcite[38–48]{Bertelli2021} warn that reputation-sourced authority can weaken democratic oversight. Media scrutiny and participation do not escape framing dependence \parencites[385–408]{Maggetti2011}[459–473]{Wood2021}. Finally, evaluations of technocratic governance are conditional and politically heterogeneous. \textcite[1008–1025]{Angelou2024} and \textcite[627–646]{Dellmuth2019} show that public responses to expert-led policies depend on how independence, competence, and performance are interpreted, while \textcite[145–163]{VanderVeer2026} document how populist contestation destabilizes support for technocratic arrangements. Policy feedback further entrenches frame dependence \parencite[111–127]{Soss2007}.

This article contributes an explicitly political account of framing in delegated healthcare rationing. We argue that framing does more than shift mass risk preferences; it structures how citizens evaluate contested state funding decisions by activating distinct standards of responsibility and obligation. When decisions are framed as loss avoidance and rescue, citizens are more likely to support funding and to assign responsibility for inaction. When decisions are framed as protecting gains and safeguarding finite collective resources, citizens are more likely to endorse restraint and cost containment. Using a population-based survey experiment embedded in the 2021 Finnish Medicines Barometer, where respondents evaluate an identical clinical vignette under alternative frames, the study shows how framing reconfigures contestation over state funding decisions in a distributive domain where delegation is intended to insulate policy from politics but instead reshapes how mass preferences are formed and expressed.



\section*{Prospect Theory, Framing, and Healthcare Policy Preferences}

Prospect theory begins from a foundational challenge to standard economic and rational-choice models of political behavior: individuals do not evaluate outcomes solely in terms of final states or expected utilities, but relative to reference points that structure whether outcomes are perceived as gains or losses. When decisions involve risk, uncertainty, and morally salient stakes, these reference points become decisive. As \textcite[263]{Kahneman1979} demonstrate, ``the psychological principles that govern the perception of decision problems and the evaluation of probabilities and outcomes produce predictable shifts of preference when the same problem is framed in different ways.’’ These shifts are systematic rather than idiosyncratic, implying that public preferences can change dramatically even when objective facts remain constant.

At the center of prospect theory lies loss aversion, the empirical regularity that losses loom larger than gains. Individuals experience the pain of loss more intensely than the pleasure of equivalent gains, generating asymmetric valuation of outcomes. As \textcite[193]{Kahneman1991a} explain, loss aversion---the disutility of giving up an object is greater that the utility associated with acquiring it.'' Similarly, \textcite[73]{K.R2024} define loss aversion as a cognitive bias where individuals are strongly motivated to avoid losses or psychologically, they perceive loss is more severe than an equivalent gain.’’ This asymmetry matters politically because it implies that policy support depends not only on what outcomes are expected, but on whether those outcomes are construed as avoiding losses or securing gains.

Prospect theory further predicts that risk preferences are context-dependent. Individuals tend to be risk averse in the domain of gains and risk seeking in the domain of losses. As \textcite[334]{Vis2011} explains, a principal feature of prospect theory is that it posits that individuals' risk tendency varies across contexts, with individuals being risk averse in the domain of gains and risk accepting in the domain of losses.'' The reference point—often, though not always, the status quo—determines which domain is activated. Empirically, \textcite[334]{Vis2011} notes that individuals use a reference point, usually the status quo, to establish whether they find themselves in a situation or domain of losses or of gains’’ and that ``losses weigh typically two to two and a half times more heavily than gains.’’

Healthcare policy provides an especially revealing arena for prospect theory because reference points are unusually fluid and normatively charged. Unlike many distributive policy domains, healthcare decisions frequently involve life-threatening risks, identifiable individuals, and profound uncertainty about outcomes. In these contexts, the relevant reference point is rarely limited to fiscal baselines or aggregate welfare. Instead, reference points are constructed around expected health trajectories and moral expectations about care. For an identifiable cancer patient with a poor prognosis, the salient reference point becomes imminent death or severe deterioration absent intervention. Relative to this reference point, any treatment offering even a small probability of benefit is framed as an opportunity to avert catastrophic loss. By contrast, for taxpayers, policymakers, and healthcare systems, the salient reference point is the maintenance of existing healthcare provision and fiscal sustainability. Relative to this reference point, allocating large sums to uncertain treatments appears as a threat to other patients and services.

These competing reference points are not inherent in the policy problem itself but are activated through framing. When a cancer case is described as a patient has exhausted all standard treatments and faces imminent death,'' attention is drawn to an individual loss that can potentially be avoided. When the same case is described as healthcare budgets are finite and funds spent here will not be available elsewhere,’’ attention shifts to system-level opportunity costs. Prospect theory predicts that these shifts in reference points will systematically alter citizens’ risk preferences, even when the underlying clinical facts and costs remain identical.

When healthcare decisions are framed in terms of potential losses—most notably, the loss of an identifiable patient’s life or health if treatment is withheld—prospect theory predicts risk-seeking behavior. In such loss-domain contexts, decision-makers are more willing to accept uncertain and costly interventions to avoid a catastrophic outcome. As \textcite[290]{McDermott2004} explains in the context of political decision-making, ``leaders in a bad situation, where things are bad or likely to get worse, are more likely to make risky choices to recover their losses.’’ Although articulated with reference to political leaders, this logic generalizes to citizens evaluating collective policy choices under conditions of perceived loss.

This mechanism provides a behavioral explanation for the rule of rescue. As \textcite[2407]{McKie2003} observe, ``when public decision-making is structured so that a decision focuses on a specific identifiable victim rather than on aggregate or statistical victims, substantial resources are sometimes devoted to rescue.’’ Prospect theory explains why such cases exert disproportionate influence: identifiable patients are framed as facing certain losses, which places citizens squarely in the loss domain. Under these conditions, risk-seeking preferences emerge, making high-cost and clinically uncertain interventions appear justified despite their inefficiency from a system-wide perspective.

Clinical and qualitative evidence reinforces this interpretation. Interviewing oncologists, patients, and family members, \textcite[1]{Bashkin2022} find that although economic considerations are acknowledged, patients expect'' clinicians to prioritize treatment possibilities. This expectation reflects an implicit loss-frame in which inaction is equated with avoidable harm. Similarly, \textcite[33]{Sinclair2022a} emphasize that rescue obligations are tied to avoiding morally salient losses at the end of life, arguing that people are particularly averse to depriving patients of opportunities to sort out their affairs, say goodbyes to family and friends, review their life, or come to terms with death itself.’’ In these accounts, loss is not limited to survival probabilities but encompasses dignity, narrative closure, and moral responsibility.

By contrast, when healthcare decisions are framed in terms of utility maximization, opportunity costs, and finite resources, prospect theory predicts risk-averse behavior. In this gains-domain framing, the reference point is the preservation of existing healthcare capacity and services. Relative to this baseline, approving an expensive and uncertain treatment constitutes a gamble that could undermine care for others. As \textcite[334]{Vis2011} predict, individuals are ``risk averse in the domain of gains,’’ preferring options that protect the status quo rather than risk losses to it.

Evidence from health policy research is consistent with this expectation. \textcite[3]{Scheijmans2025} show that when Dutch citizens are reminded that there are limited resources for healthcare,'' they evaluate expensive medicines more stringently, with an unfavourable cost-benefit ratio’’ emerging as the principal reason for opposing reimbursement. In this frame, citizens emphasize system sustainability and demand higher certainty of benefit before accepting costly interventions. Comparable patterns appear in other policy areas. \textcite[1]{Svenningsen2021} demonstrate that ``a gain and loss framing influence social preferences for the distributional outcomes of climate policy,’’ with loss frames generating higher willingness to bear costs than gain frames. Extending this insight to healthcare implies that when system maintenance is framed as a gain to be protected, support for risky spending declines.

Healthcare decisions further complicate prospect theory because outcomes are inherently uncertain. Citizens evaluating expensive cancer medicines must assess not only clear financial costs but also ambiguous clinical evidence. When treatments offer uncertain benefits—as \textcite[1]{Aziz2020} document for many oncology drugs—citizens face what can be described as a nested reference-point problem. They must weigh potential health losses against resource losses under uncertainty. Prospect theory predicts that under loss framing, uncertainty about benefits does not suppress support; instead, individuals become more willing to gamble on low-probability gains to avoid catastrophic loss. Under gains framing, the same uncertainty reinforces risk aversion, leading to demands for stronger evidence of effectiveness.

This logic helps explain why cost-effectiveness analysis so often fails to persuade publics confronted with life-extending cancer care. Cost-effectiveness analysis is inherently gains-framed: it evaluates whether resources should be allocated to maximize aggregate health benefits relative to a system-level reference point. As \textcite[851]{Aziz2020} illustrate, conclusions that ``adding atezolizumab to nab-paclitaxel resulted in an additional 0.361 QALYs at an ICER of S\$324,550 per QALY gained’’ implicitly frame the decision as one of optimal resource allocation. Prospect theory predicts that in such gains-domain contexts, citizens will exhibit risk aversion and resist high-cost, uncertain interventions.

When the same treatment is instead presented through a rescue narrative—``a specific patient has cancer, standard treatments have failed, and without this medicine the patient will die’’—the reference point shifts decisively. The decision becomes one of loss avoidance rather than efficiency. Under this loss-domain framing, prospect theory predicts risk-seeking behavior, making support for expensive treatment more likely despite unfavorable cost-effectiveness. Public rejection of cost-effectiveness arguments thus reflects not ignorance or inconsistency, but a mismatch between the reference points assumed by technical analyses and those activated by morally salient frames.

The broader implication is that the tension between individual rescue and population health is not simply a clash of values, but a systematic consequence of framing. Individual rescue cases activate loss-domain reference points centered on identifiable mortality, while population-health and stewardship frames activate gains-domain reference points centered on system sustainability. As \textcite[290]{McDermott2004} argue, prospect theory offers a number of advantages that justify the use of psychological models over alternative models of political behavior'' because it emphasizes the importance of loss in calculations of value and utility.’’ Citizens’ apparent inconsistency—supporting both fiscal restraint and costly rescue—is therefore predictable rather than paradoxical.

By integrating prospect theory with the literature on the rule of rescue, this framework clarifies how framing structures public preferences in healthcare policy. Neither rescue logic nor cost-effectiveness reasoning alone can fully account for mass opinion in welfare states. Instead, recognizing how loss and gain frames activate distinct reference points explains why public support oscillates between compassion-driven risk acceptance and efficiency-driven restraint. The theory developed here thus provides a behavioral foundation for understanding healthcare politics as a domain in which moral imperatives and distributive constraints are mediated through systematic, frame-dependent patterns of risk evaluation.


\section{Empirical Section}

The empirical analysis draws on data from the 2021 Finnish Medicines Barometer, a national, cross-sectional population survey administered biennially by the Finnish Medicines Agency to examine experiences, opinions, and values related to health, medicines, and well-being. The 2021 wave included an ad hoc experimental module designed specifically to assess public attitudes toward the public funding of novel, high-cost oncology medicines characterized by uncertain clinical benefit. Finland provides a particularly appropriate research context for examining these questions. The Finnish healthcare system is comprehensive and publicly funded, closely resembling other advanced welfare states in which questions of healthcare rationing and legitimacy are politically salient. Moreover, Nordic healthcare systems face precisely the pressures that motivate this study. As \textcite[1216--1222]{Torkki2022} document, ``cancer care in Nordic countries has significant differences in both cost structures and in the development of cost drivers,'' driven in substantial part by expensive pharmaceutical innovations with uncertain value.

Data collection was carried out by a professional market research company (Taloustutkimus Ltd) using a pre-recruited online panel of approximately 40,000 Finnish citizens. To reach the target sample size, 10,105 invitations were distributed to panel members. The sample was stratified to ensure balance across gender, age, education level, and geographic region, with eligibility restricted to Finnish-speaking citizens aged 18--79 years. The final analytic sample consists of 2,081 respondents, corresponding to a completion rate of approximately 20.6 percent. While online panels are not probability samples, stratification and quota-based recruitment ensure close correspondence with the Finnish adult population on key sociodemographic dimensions, making the data suitable for inference about mass public opinion in a welfare-state context.

The experimental module employed a randomized between-subjects design focused on framing effects. Although the module included both an information experiment and a framing experiment, the analysis presented here focuses exclusively on the framing experiment. Respondents were randomly assigned to one of three experimental conditions—Scenarios B, C, and D—each administered to approximately 500 respondents. All respondents received an identical clinical case description and decision-making task; the experimental manipulation consisted solely of the framing information presented between the case description and the decision task. This design ensures that any observed differences in preferences across conditions can be attributed to framing rather than to differences in substantive information.

The clinical case described a new medicine indicated for the treatment of a specific incurable cancer. The description emphasized several features that mirror real-world oncology policy decisions. Respondents were informed that the medicine does not cure the disease, that laboratory test results suggest it destroys cancer cells in approximately one-third of patients who receive it, and that it remains unknown whether the medicine extends patients’ lives or improves quality of life relative to existing treatment options. The case further specified that the medicine carries a substantial risk of adverse effects and would be prescribed to a very small patient population—approximately 10 to 20 patients annually in Finland. Importantly, respondents were told that if approved for public funding, the medicine would cost more than 60,000 euros per patient at the final stage of cancer treatment. This cost level reflects the magnitude of expenditures associated with innovative cancer medicines and is consistent with empirical estimates reported in the health economics literature (see \textcite[1]{Aziz2020}). By combining high cost, uncertain benefit, and limited patient eligibility, the case captures the core elements of contemporary cancer drug reimbursement dilemmas.

Following the case description, respondents in the control condition (Scenario B) proceeded directly to the decision-making task without additional framing, receiving only the clinical and cost information. In contrast, respondents in the two treatment conditions received additional framing statements designed to shift the reference point through which the decision was evaluated. In the loss-framed condition (Scenario C), respondents were told: ``There is no cure for this particular type of cancer. The new medicine is a possible option for patients who have already received multiple treatments and for whom the remaining options are limited.'' This framing emphasizes the absence of alternatives and the patient’s proximity to death, thereby activating a loss-domain reference point centered on the avoidable loss of life. In the gains-framed condition (Scenario D), respondents were told: ``The funds available for healthcare are finite. The adoption of the new medicine means that the funds used to pay for it will not be available elsewhere in healthcare.'' This framing highlights opportunity costs and system-level trade-offs, activating a gains-domain reference point focused on protecting existing healthcare provision. Both frames reflect language commonly used in real policy debates and are theoretically motivated by prospect theory’s emphasis on reference-point-dependent risk preferences.

After exposure to the case description and, where applicable, the framing manipulation, all respondents completed the same decision-making task. They were asked: ``What kind of decision regarding the new medicine’s use would you find acceptable? Please choose the option that best reflects your opinion.'' Four response options were provided: unconditional public funding regardless of price; conditional funding contingent on price reductions; rejection of public funding; and an explicit ``I don’t know'' option. These response categories capture meaningful variation in attitudes toward risk and cost in healthcare spending. Unconditional funding represents the most risk-seeking option, accepting both high financial cost and uncertain clinical benefit. Conditional funding reflects moderate risk aversion, accepting treatment only if financial exposure is reduced. Rejection represents the most risk-averse position, prioritizing resource protection over uncertain benefit. The inclusion of an explicit uncertainty option allows respondents to express ambivalence rather than forcing artificial choice.

Prospect theory yields clear expectations for how preferences should vary across framing conditions. When the decision context is framed as loss avoidance—emphasizing the imminent death of identifiable patients and the absence of alternatives—citizens are expected to become more risk-seeking, increasing support for unconditional funding and reducing outright rejection. When the same decision is framed in terms of protecting gains—emphasizing finite budgets and opportunity costs—citizens are expected to exhibit greater risk aversion, shifting support toward conditional funding or rejection. Because the clinical information and decision options are held constant across conditions, observed differences in preferences can be interpreted as evidence of framing-induced shifts in reference points and risk orientation rather than changes in substantive beliefs about the medicine itself.



\begin{knitrout}
\definecolor{shadecolor}{rgb}{0.969, 0.969, 0.969}\color{fgcolor}\begin{figure}

{\centering \includegraphics[width=\maxwidth]{figure/pred-1} 

}

\caption[Framing Effects on Public Support for Funding a High-Cost Cancer Medicine]{Framing Effects on Public Support for Funding a High-Cost Cancer Medicine}\label{fig:pred}
\end{figure}

\end{knitrout}






\FloatBarrier
\newpage



\clearpage
\newpage
\pagenumbering{roman}
\setcounter{page}{1}
\printbibliography
\clearpage
\newpage



%%%%%%%%%%%%%%%%%%%%%%%%%%%%%%%%%%%%%%%%%%%%%%
% WORD COUNT
%%%%%%%%%%%%%%%%%%%%%%%%%%%%%%%%%%%%%%%%%%%%%%
\clearpage



\begin{center}
\vspace*{\stretch{1}}
\dotfill
\dotfill {\huge {\bf Word count}: 4,365} \dotfill
\dotfill
\vspace*{\stretch{1}}
\end{center}

\clearpage

%%%%%%%%%%%%%%%%%%%%%%%%%%%%%%%%%%%%%%%%%%%%%%
% WORD COUNT
%%%%%%%%%%%%%%%%%%%%%%%%%%%%%%%%%%%%%%%%%%%%%%


% Online Appendix
%\newpage
%\section{Online Appendix}
%\pagenumbering{Roman}
%\setcounter{page}{1}



%% reset tables and figures counter
\setcounter{table}{0}
\renewcommand{\thetable}{A\arabic{table}}
\setcounter{figure}{0}
\renewcommand{\thefigure}{A\arabic{figure}}



\section{Appendix}
\pagenumbering{roman}
\setcounter{page}{1}


\subsection{Regression Models}
\FloatBarrier

\begin{table}[!htbp]
\centering
\caption{Framing Effects on Public Support for Funding Expensive Cancer Medicines}
\label{tab:ologit-models}
%\setlength{\tabcolsep}{3.5pt} % tighten intercolumn spacing
\begin{tabular}{ll}
\hline
& Ordinal logit \\ \hline
Rule of rescue (vs. control) & \num{1.478}*** \\
& [\num{1.252}, \num{1.746}] \\
Utility maximizing (vs. control) & \num{0.762}** \\
& [\num{0.644}, \num{0.901}] \\
Male (vs. female) & \num{0.953} \\
& [\num{0.830}, \num{1.095}] \\
Age & \num{1.016}*** \\
& [\num{1.012}, \num{1.021}] \\
Income: middle (vs. low) & \num{0.858} \\
& [\num{0.719}, \num{1.023}] \\
Income: high (vs. low) & \num{0.944} \\
& [\num{0.778}, \num{1.147}] \\
Income: other/unknown & \num{0.821} \\
& [\num{0.652}, \num{1.035}] \\
Eligible for Kela reimbursement & \num{0.977} \\
& [\num{0.829}, \num{1.151}] \\
Num.Obs. & \num{2460} \\
\hline
\end{tabular}

\par\vspace{0.4ex}
\footnotesize \emph{Notes:} Entries report odds ratios from an ordinal logistic regression. The dependent variable measures respondents' preferred funding decision for a novel, high-cost cancer medicine, ordered from unconditional public funding to outright rejection. The key independent variable captures experimental framing of the decision context. Control variables include gender, age, income group, education, eligibility for Kela medicine reimbursement, and self-reported medicine expenditure. Models are estimated using survey weights. 90\% confidence intervals in parentheses. $^{*}p<0.10$, $^{**}p<0.05$, $^{***}p<0.01$.
\end{table}



\end{document}





