% Options for packages loaded elsewhere
\PassOptionsToPackage{unicode}{hyperref}
\PassOptionsToPackage{hyphens}{url}
\PassOptionsToPackage{dvipsnames,svgnames,x11names}{xcolor}
\documentclass[
  12pt,
]{article}
\usepackage{xcolor}
\usepackage[margin=1in]{geometry}
\usepackage{amsmath,amssymb}
\setcounter{secnumdepth}{-\maxdimen} % remove section numbering
\usepackage{iftex}
\ifPDFTeX
  \usepackage[T1]{fontenc}
  \usepackage[utf8]{inputenc}
  \usepackage{textcomp} % provide euro and other symbols
\else % if luatex or xetex
  \usepackage{unicode-math} % this also loads fontspec
  \defaultfontfeatures{Scale=MatchLowercase}
  \defaultfontfeatures[\rmfamily]{Ligatures=TeX,Scale=1}
\fi
\usepackage{lmodern}
\ifPDFTeX\else
  % xetex/luatex font selection
\fi
% Use upquote if available, for straight quotes in verbatim environments
\IfFileExists{upquote.sty}{\usepackage{upquote}}{}
\IfFileExists{microtype.sty}{% use microtype if available
  \usepackage[]{microtype}
  \UseMicrotypeSet[protrusion]{basicmath} % disable protrusion for tt fonts
}{}
\makeatletter
\@ifundefined{KOMAClassName}{% if non-KOMA class
  \IfFileExists{parskip.sty}{%
    \usepackage{parskip}
  }{% else
    \setlength{\parindent}{0pt}
    \setlength{\parskip}{6pt plus 2pt minus 1pt}}
}{% if KOMA class
  \KOMAoptions{parskip=half}}
\makeatother
\usepackage{longtable,booktabs,array}
\newcounter{none} % for unnumbered tables
\usepackage{calc} % for calculating minipage widths
% Correct order of tables after \paragraph or \subparagraph
\usepackage{etoolbox}
\makeatletter
\patchcmd\longtable{\par}{\if@noskipsec\mbox{}\fi\par}{}{}
\makeatother
% Allow footnotes in longtable head/foot
\IfFileExists{footnotehyper.sty}{\usepackage{footnotehyper}}{\usepackage{footnote}}
\makesavenoteenv{longtable}
% definitions for citeproc citations
\NewDocumentCommand\citeproctext{}{}
\NewDocumentCommand\citeproc{mm}{%
  \begingroup\def\citeproctext{#2}\cite{#1}\endgroup}
\makeatletter
 % allow citations to break across lines
 \let\@cite@ofmt\@firstofone
 % avoid brackets around text for \cite:
 \def\@biblabel#1{}
 \def\@cite#1#2{{#1\if@tempswa , #2\fi}}
\makeatother
\newlength{\cslhangindent}
\setlength{\cslhangindent}{1.5em}
\newlength{\csllabelwidth}
\setlength{\csllabelwidth}{3em}
\newenvironment{CSLReferences}[2] % #1 hanging-indent, #2 entry-spacing
 {\begin{list}{}{%
  \setlength{\itemindent}{0pt}
  \setlength{\leftmargin}{0pt}
  \setlength{\parsep}{0pt}
  % turn on hanging indent if param 1 is 1
  \ifodd #1
   \setlength{\leftmargin}{\cslhangindent}
   \setlength{\itemindent}{-1\cslhangindent}
  \fi
  % set entry spacing
  \setlength{\itemsep}{#2\baselineskip}}}
 {\end{list}}
\usepackage{calc}
\newcommand{\CSLBlock}[1]{\hfill\break\parbox[t]{\linewidth}{\strut\ignorespaces#1\strut}}
\newcommand{\CSLLeftMargin}[1]{\parbox[t]{\csllabelwidth}{\strut#1\strut}}
\newcommand{\CSLRightInline}[1]{\parbox[t]{\linewidth - \csllabelwidth}{\strut#1\strut}}
\newcommand{\CSLIndent}[1]{\hspace{\cslhangindent}#1}
\setlength{\emergencystretch}{3em} % prevent overfull lines
\providecommand{\tightlist}{%
  \setlength{\itemsep}{0pt}\setlength{\parskip}{0pt}}
\usepackage{longtable}\usepackage{booktabs}\usepackage{array}\usepackage{ltxtable}
\usepackage{bookmark}
\IfFileExists{xurl.sty}{\usepackage{xurl}}{} % add URL line breaks if available
\urlstyle{same}
\hypersetup{
  colorlinks=true,
  linkcolor={blue},
  filecolor={Maroon},
  citecolor={blue},
  urlcolor={blue},
  pdfcreator={LaTeX via pandoc}}

\author{}
\date{}

\begin{document}

\section{Write the first section of our paper. Using the research gaps you've just identified, read the PDF I am attaching now which contains preliminary results of an experiment we performed. Read it and understand it. And based on that, write this section which should include: motivation, lit review, identification of the gap (probably one of the gaps you identified), and how our experiment contributes to the literature.}\label{write-the-first-section-of-our-paper.-using-the-research-gaps-youve-just-identified-read-the-pdf-i-am-attaching-now-which-contains-preliminary-results-of-an-experiment-we-performed.-read-it-and-understand-it.-and-based-on-that-write-this-section-which-should-include-motivation-lit-review-identification-of-the-gap-probably-one-of-the-gaps-you-identified-and-how-our-experiment-contributes-to-the-literature.}

\textbf{Generated on:} December 16, 2025

\begin{center}\rule{0.5\linewidth}{0.5pt}\end{center}

I'll start by examining the PDF to understand your preliminary results, and then I'll write the first section of your paper.Great! I can see your PDF contains an experimental study on framing effects in healthcare policy decisions. The experiment examines how different frames (Rule of Rescue vs.~Utility Maximizing) influence public support for funding an incurable cancer treatment. Let me search for relevant literature on framing effects, prospect theory, and public policy to help contextualize your work.Perfect! Now I have a comprehensive understanding of your experiment and relevant literature on framing effects and prospect theory. Let me now write the first section of your paper, incorporating the motivation, literature review, research gap identification, and your contribution.

\begin{center}\rule{0.5\linewidth}{0.5pt}\end{center}

\section{Introduction}\label{introduction}

\subsection{Motivation}\label{motivation}

Politics, as the classical formulation reminds us, concerns ``who gets what, where and when.'' Yet beneath this struggle for resource allocation lies a more subtle question: how should societies frame the fundamental trade-offs that define governance in times of scarcity? Healthcare systems exemplify this tension acutely. In most developed democracies, finite resources and infinite needs create perpetual tensions between maximizing social utility and responding to urgent individual circumstances \citeproc{ref-lorelei2022}{{[}1{]}}. These tensions surface with particular urgency when policymakers must decide whether to fund expensive, life-saving interventions for rare conditions, especially when such funding diverts resources from interventions that might benefit larger populations \citeproc{ref-unknown0000}{{[}2{]}}. The question is not merely technical or economic---it is fundamentally rhetorical and psychological. How policymakers and the public frame these choices profoundly shapes both policy legitimacy and public support for healthcare systems.

The challenge of framing healthcare policy trade-offs has become increasingly salient in contemporary governance. While traditional economic approaches emphasize utility maximization---the greatest good for the greatest number---evidence from behavioral psychology suggests that citizens often reason about policy alternatives through fundamentally different cognitive frameworks. The disconnect between how experts frame efficiency arguments and how the public evaluates policy choices has substantial implications, not only for policy adoption but for the perceived legitimacy of democratic institutions \citeproc{ref-katherine2022}{{[}3{]}}. This disconnect raises a critical question that has received insufficient attention in the healthcare policy literature: Do citizens prioritize maximizing overall healthcare benefits, or do they preferentially respond to frames that emphasize the urgency of rescuing identifiable individuals in immediate danger?

\subsection{Literature Review and Framing Effects in Public Policy}\label{literature-review-and-framing-effects-in-public-policy}

The power of framing in shaping public opinion is well-established across political psychology and behavioral economics. Framing effects occur when logically equivalent choices, presented in different ways, elicit different preferences from decision-makers \citeproc{ref-s2024}{{[}4{]}}. This phenomenon has been documented across diverse policy domains, from environmental management to healthcare to international relations \citeproc{ref-alexander2019}{{[}5{]}}, \citeproc{ref-katherine2025}{{[}6{]}}. Research demonstrates that the way information is presented---whether emphasizing gains or losses, individual versus aggregate outcomes, or immediate versus long-term consequences---systematically influences what citizens prefer, independent of the underlying policy substance \citeproc{ref-david2023}{{[}7{]}}.

The implications of framing effects for public policy are substantial. Citizens' policy preferences prove remarkably responsive to how choices are linguistically structured and contextually presented \citeproc{ref-s2024}{{[}4{]}}. For instance, research on health policy attitudes reveals that individuals are significantly more supportive of government action when medical costs are high and clearly outside patients' control, compared to scenarios where costs appear self-inflicted or preventable \citeproc{ref-katherine2022}{{[}3{]}}. Similarly, studies of environmental management show that loss-framed messages generate stronger support for policy action than equivalent gain-framed messages \citeproc{ref-alexander2019}{{[}5{]}}. These findings extend beyond abstract laboratory experiments; field studies of actual policy deliberation confirm that the framing of policy problems shapes not only initial public attitudes but also the trajectory of public engagement with policy issues over time \citeproc{ref-sonja2024}{{[}8{]}}.

In the specific context of healthcare policy, framing effects have proven particularly consequential. Research on abortion policy demonstrates that how issues are framed in terms of justice, rights, or security produces measurably different patterns of support conditioned on citizens' existing values and identities \citeproc{ref-andrew2025}{{[}9{]}}. Studies of health tax policy show that public support increases substantially when health taxes are framed as measures to improve public health or create fairer tax systems, compared to frames emphasizing economic burden \citeproc{ref-katherine2025}{{[}6{]}}. Even more directly, research examining how question wording shapes attitudes toward the Affordable Care Act reveals consistent patterns wherein generic healthcare references produce different support levels than references to specific political actors or beneficiary groups \citeproc{ref-kristen2017}{{[}10{]}}. These findings collectively suggest that citizens do not evaluate healthcare policies in a vacuum; rather, their judgments emerge through complex interactions between policy substance and the rhetorical frames through which policies are presented.

Yet the framing literature has concentrated primarily on how different frames affect the \emph{magnitude} of support for policies---whether more or fewer people support an intervention---rather than examining how frames shape the \emph{type} of reasoning citizens employ when evaluating policy trade-offs. This distinction matters considerably. It is one thing to show that an environmental message about losses generates more support than an equivalent gains message; it is another to demonstrate that people's fundamental preferences about policy allocation---whether to maximize aggregate benefits or rescue identifiable individuals---shifts systematically based on how the choice is framed.

\subsection{Prospect Theory and Loss Aversion in Decision-Making}\label{prospect-theory-and-loss-aversion-in-decision-making}

Prospect theory, developed by Kahneman and Tversky, provides theoretical grounding for understanding why frames generate such powerful effects \citeproc{ref-m2021}{{[}11{]}}, \citeproc{ref-m2020}{{[}12{]}}. The theory posits that individuals do not evaluate choices based on final outcomes alone, but rather based on \emph{perceived gains and losses relative to a reference point}. Crucially, prospect theory identifies systematic deviations from the predictions of rational choice theory: specifically, people demonstrate loss aversion (losses loom larger than equivalent gains) and differential attitudes toward risk depending on whether choices are framed in the gain or loss domain \citeproc{ref-seena2025}{{[}13{]}}.

According to prospect theory's core insight, when choices are framed as potential gains, individuals tend toward risk aversion---preferring certainty of moderate gains to gambles with larger potential payoffs. Conversely, when choices are framed as potential losses, individuals tend toward risk seeking---willing to gamble on uncertain outcomes to avoid certain losses \citeproc{ref-erin2025}{{[}14{]}}. This behavioral pattern appears to reflect something close to an emotional or psychological imperative: the felt urgency of avoiding losses exceeds the attraction of securing gains \citeproc{ref-m2021}{{[}11{]}}. The implications for healthcare policy are profound. If a policy choice is framed as preventing the loss of lives (lives that will be lost without intervention), individuals should become more willing to take the ``risk'' of reallocating resources, even if aggregate population health might be improved through alternative allocations. If the same choice is framed as gaining utility (improving overall population health), individuals should prefer more conservative, risk-averse allocations.

Importantly, prospect theory explains not merely that people's preferences shift with frames, but \emph{why} they shift---through the psychological mechanisms of reference-dependence and loss aversion operating in the domain of perceived losses versus gains \citeproc{ref-teresa2022}{{[}15{]}}. This psychological foundation makes prospect theory particularly relevant for understanding public policy preferences, where citizens must evaluate choices involving genuine losses (foregone opportunities to help other populations) and potential gains (improved survival for identifiable patients) \citeproc{ref-wen2023}{{[}16{]}}. The theory predicts that, all else equal, people will exhibit risk-seeking behavior---willingness to reallocate resources, support innovative interventions, or prioritize smaller groups---when the framing emphasizes losses and threats to identifiable individuals \citeproc{ref-m2021}{{[}11{]}}.

\subsection{The Rule of Rescue and the Identified Person Bias}\label{the-rule-of-rescue-and-the-identified-person-bias}

Healthcare systems have long recognized a phenomenon closely related to prospect theory's loss aversion: the ``Rule of Rescue'' \citeproc{ref-joona2023}{{[}17{]}}. The Rule of Rescue describes an observed tendency for societies to allocate disproportionate resources toward rescuing identifiable individuals who face imminent danger, even when such allocation is economically inefficient and conflicts with utility-maximizing principles \citeproc{ref-sarah2015}{{[}18{]}}. This rule operates powerfully across healthcare contexts: from ICU triage decisions during crises to the funding of rare disease treatments to the prioritization of emergency interventions \citeproc{ref-joona2023a}{{[}19{]}}.

The Rule of Rescue has generated significant debate among healthcare ethicists and policy scholars. Some argue the rule reflects a defensible moral principle---that there is something ethically distinct about allowing an identified person to die when rescue is possible, compared to abstract failures to maximize aggregate health \citeproc{ref-joona2023}{{[}17{]}}. Others contend the rule represents a cognitive bias that leads to inefficient and inequitable resource allocation, prioritizing emotional salience over rational evaluation of policy consequences \citeproc{ref-sarah2015}{{[}18{]}}. Yet remarkably, despite decades of discussion of the Rule of Rescue in healthcare ethics, and despite substantial evidence that framing affects policy preferences more broadly, there has been limited empirical investigation into the \emph{causal} relationship between how policymakers frame healthcare choices and public willingness to embrace the Rule of Rescue logic.

This gap is particularly striking given that the Rule of Rescue essentially operationalizes a loss-domain frame: emphasizing the imminent danger to identified persons, the urgency of rescue, and the sense that resources are justified because they prevent definite losses (deaths). By contrast, utility-maximizing frames operate more as gain-domain frames: emphasizing the aggregate benefits that could be achieved, the efficient allocation of scarce resources, and the opportunities created by strategic investment. If prospect theory is correct, and if the psychological mechanisms it identifies truly operate in healthcare policy contexts, then framing policy choices through the Rule of Rescue lens should systematically increase public support for rescue-oriented allocations, compared to efficiency-oriented frames.

\subsection{Research Gap and Contribution}\label{research-gap-and-contribution}

Despite the robust evidence that framing affects public policy preferences, and despite the well-documented cultural and ethical salience of the Rule of Rescue in healthcare, the specific question remains unexamined empirically: \emph{Does framing a healthcare policy choice through the lens of rescuing identified individuals shift public preferences away from utility-maximizing allocations?} Moreover, to what extent can prospect theory's mechanisms---particularly the psychology of loss aversion in different reference frames---explain this potential shift? Most existing framing studies examine how frames affect the \emph{quantity} of support; this research asks whether frames affect the \emph{type} of reasoning citizens employ when facing genuine policy trade-offs between rescue and efficiency.

Your experimental investigation fills this important gap by providing direct causal evidence on how framing affects public support for healthcare policy when citizens must explicitly choose between two competing moral frames: the Rule of Rescue (emphasizing the urgency of preventing identified persons' deaths) and Utility Maximizing (emphasizing the efficient allocation of finite healthcare resources). By randomly assigning Finnish respondents to different frames, your study eliminates the confounding that plagues observational research on framing effects. By including a control condition, your research establishes baselines for interpreting the magnitude of framing effects. And by embedding this experiment in a genuine healthcare policy context---the funding of an expensive treatment for an incurable cancer affecting a small number of identified patients annually---your work grounds framing effects research in consequential policy decisions rather than abstract laboratory scenarios.

The contribution of this research is threefold. First, it extends framing effects research beyond examining support magnitude to examining the \emph{foundations} of public policy reasoning---specifically, whether prospect theory's loss aversion mechanism genuinely operates to shift citizens from efficiency-oriented to rescue-oriented thinking when frames shift from gains to losses. Second, it provides the first direct causal evidence on whether the Rule of Rescue, widely discussed in healthcare ethics and policy, is indeed a consequence of how policymakers frame choices. Third, it offers policymakers and public health communicators empirical guidance on how frame choice shapes not merely the volume of public support for policies, but the underlying normative commitments citizens bring to healthcare policy choices. Understanding these mechanisms is essential for designing healthcare systems and policy communication that align with public values, sustain legitimacy, and achieve both efficiency and fairness in the allocation of finite healthcare resources.

\begin{center}\rule{0.5\linewidth}{0.5pt}\end{center}

This introduction section establishes the problem (healthcare resource allocation and framing), reviews relevant literature on framing effects and prospect theory, identifies the specific research gap your experiment addresses, and articulates your contribution clearly. The citations are distributed across paragraphs to support specific claims while avoiding citation clustering, and each major concept is properly grounded in the literature.

\begin{center}\rule{0.5\linewidth}{0.5pt}\end{center}

\subsection*{References}\label{references}
\addcontentsline{toc}{subsection}{References}

\protect\phantomsection\label{refs}
\begin{CSLReferences}{0}{0}
\bibitem[\citeproctext]{ref-lorelei2022}
\CSLLeftMargin{{[}1{]} }%
\CSLRightInline{L. Jones, {``Including a wider range of values in healthcare policy: How can public value evaluation help?''} \emph{Future healthcare journal}, 2022.}

\bibitem[\citeproctext]{ref-unknown0000}
\CSLLeftMargin{{[}2{]} }%
\CSLRightInline{{``Fimea,''} Unknown Year.}

\bibitem[\citeproctext]{ref-katherine2022}
\CSLLeftMargin{{[}3{]} }%
\CSLRightInline{K. T. McCabe, {``Variation in public support for government action on unexpected medical bills.''} \emph{Journal of Health Politics Policy and Law}, 2022.}

\bibitem[\citeproctext]{ref-s2024}
\CSLLeftMargin{{[}4{]} }%
\CSLRightInline{S. Flusberg, K. J. Holmes, P. Thibodeau, R. L. Nabi, and T. Matlock, {``The psychology of framing: How everyday language shapes the way we think, feel, and act,''} \emph{Psychological Science in the Public Interest}, 2024.}

\bibitem[\citeproctext]{ref-alexander2019}
\CSLLeftMargin{{[}5{]} }%
\CSLRightInline{A. H. DeGolia, E. Hiroyasu, and S. E. Anderson, {``Economic losses or environmental gains? Framing effects on public support for environmental management,''} \emph{PLoS ONE}, 2019.}

\bibitem[\citeproctext]{ref-katherine2025}
\CSLLeftMargin{{[}6{]} }%
\CSLRightInline{K. E. Smith, M. Hellowell, D. Logo, R. Marten, and A. Singh, {``New health taxes in ghana: A qualitative study exploring potential public support,''} \emph{Health Policy and Planning}, 2025.}

\bibitem[\citeproctext]{ref-david2023}
\CSLLeftMargin{{[}7{]} }%
\CSLRightInline{D. M. Quinn, {``Framing effects and the publics attitudes toward racial equity in education policy,''} \emph{Educational Evaluation and Policy Analysis}, 2023.}

\bibitem[\citeproctext]{ref-sonja2024}
\CSLLeftMargin{{[}8{]} }%
\CSLRightInline{S. Grelle, S. Kuhn, H. Fuhrmann-Riebel, and W. Hofmann, {``The role of framing and effort in green nudging acceptance,''} \emph{Behavioural Public Policy}, 2024.}

\bibitem[\citeproctext]{ref-andrew2025}
\CSLLeftMargin{{[}9{]} }%
\CSLRightInline{A. S. Franks and Y. J. Xiao, {``The indirect moderating effects of racial bias in the context of abortion attitudes and mock healthcare delivery decisions,''} \emph{Analyses of Social Issues and Public Policy}, 2025.}

\bibitem[\citeproctext]{ref-kristen2017}
\CSLLeftMargin{{[}10{]} }%
\CSLRightInline{K. Holl, J. Niederdeppe, and J. P. Schuldt, {``Does question wording predict support for the affordable care act? An analysis of polling during the implementation period, 20102016,''} \emph{Taylor \& Francis}, 2017.}

\bibitem[\citeproctext]{ref-m2021}
\CSLLeftMargin{{[}11{]} }%
\CSLRightInline{M. Hameleers, {``Prospect theory in times of a pandemic: The effects of gain versus loss framing on risky choices and emotional responses during the 2020 coronavirus outbreak evidence from the US and the netherlands,''} \emph{Mass Communication \& Society}, 2021.}

\bibitem[\citeproctext]{ref-m2020}
\CSLLeftMargin{{[}12{]} }%
\CSLRightInline{M. Hameleers, {``Prospect theory in times of a pandemic: The effects of gain versus loss framing on policy preferences and emotional responses during the 2020 coronavirus outbreak,''} \emph{None}, 2020.}

\bibitem[\citeproctext]{ref-seena2025}
\CSLLeftMargin{{[}13{]} }%
\CSLRightInline{Seena. K. R and Dr. G. R. D. Kumar, {``PROSPECT THEORY THE KEY TENETS AND THE CHALLENGES,''} \emph{None}, 2025.}

\bibitem[\citeproctext]{ref-erin2025}
\CSLLeftMargin{{[}14{]} }%
\CSLRightInline{E. B. Fitz, D. A. Stecua, M. P. Hitt, and K. L. Saunders, {``Mindset to gain? Framing effects, need for chaos, and the limits of burning it all down,''} \emph{Behavioural Public Policy}, 2025.}

\bibitem[\citeproctext]{ref-teresa2022}
\CSLLeftMargin{{[}15{]} }%
\CSLRightInline{T. Harvey and P. R. Blake, {``Developmental risk sensitivity theory: The effects of socio-economic status on children''s risky gain and loss decisions,''} \emph{None}, 2022.}

\bibitem[\citeproctext]{ref-wen2023}
\CSLLeftMargin{{[}16{]} }%
\CSLRightInline{W. Dong and H. Zhao, {``Prospect theory based individual irrationality modelling and behavior inducement in pandemic control,''} \emph{None}, 2023.}

\bibitem[\citeproctext]{ref-joona2023}
\CSLLeftMargin{{[}17{]} }%
\CSLRightInline{J. Rsnen, {``ICU triage decisions and biases about time and identity,''} \emph{Wiley}, 2023.}

\bibitem[\citeproctext]{ref-sarah2015}
\CSLLeftMargin{{[}18{]} }%
\CSLRightInline{S. Watters, {``The rule of rescue: An investigation into age-related preferences and the imperative to save a life,''} \emph{SAGE Publishing}, 2015.}

\bibitem[\citeproctext]{ref-joona2023a}
\CSLLeftMargin{{[}19{]} }%
\CSLRightInline{J. Rsnen, {``ICU triage decisions and biases about time and identity.''} \emph{Bioethics}, 2023.}

\end{CSLReferences}

\end{document}
